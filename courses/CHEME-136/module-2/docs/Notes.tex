\documentclass[11pt]{article}

% load some asm stuff -
\usepackage{amssymb}
\usepackage{amsmath}
\usepackage{amsthm}
%\usepackage{palatino,lettrine}
\usepackage{fancyhdr}
\usepackage{epsfig}
\usepackage[square,sort,comma,numbers]{natbib}
\usepackage{simplemargins}
\usepackage{setspace}
\usepackage{wrapfig}
\usepackage{hyperref}
%\usepackage{boiboites}
\usepackage[margin=0pt,font=small,labelfont=bf]{caption}
\newcommand{\boldindex}[1]{\textbf{\hyperpage{#1}}}
\usepackage{makeidx}\makeindex
\bibliographystyle{plos2015}

\usepackage{algpseudocode}
\usepackage{algorithm}


% Set the size
%\textwidth = 6.75 in
%\textheight = 9.75 in
%\oddsidemargin = 0.0 in
%\evensidemargin = 0.0 in
%\topmargin = 0.01 in
%\headheight = 0.0 in
%\headsep = 0.25 in
%\parskip = 0.15in
% \doublespace
\setallmargins{1in}

\newtheorem{example}{Example}[section]
\newtheorem{thm}{Theorem}[section]
\newtheorem{property}{Property}[section]

\theoremstyle{definition}
\newtheorem{defn}[thm]{Definition}

\makeatletter
% \renewcommand\subsection{\@startsection
% 	{subsection}{2}{0mm}
% 	{-0.05in}
% 	{0.05\baselineskip}
% 	{\normalfont\normalsize\bfseries}}
\renewcommand\subsubsection{\@startsection
	{subsubsection}{2}{0mm}
	{-0.05in}
	{-0.5\baselineskip}
	{\normalfont\normalsize\itshape\bfseries}}
\renewcommand\paragraph{\@startsection
	{paragraph}{2}{0mm}
	{-0.05in}
	{-0.5\baselineskip}
	{\normalfont\normalsize\itshape}}
\makeatother
\linespread{1.1}

\fancypagestyle{proposal}{\fancyhf{}%
	\fancyhead[RO,LE]{\thepage}%
	\fancyhead[LO,RE]{CHEME 134 Module 1 American Contract Dynamics}%
	\renewcommand\headrulewidth{1pt}}
\pagestyle{proposal}

\usepackage{mdframed}
\definecolor{lgray}{rgb}{0.92,0.92,0.92}
\definecolor{antiquewhite}{rgb}{0.98,0.92,0.84}
\definecolor{lightskyblue}{rgb}{0.93,0.95,0.99}

% defn environment
\mdfdefinestyle{theoremstyle}{% 
    linecolor=black,linewidth=1pt,% 
    frametitlerule=true,% 
    frametitlebackgroundcolor=lgray, 
    innertopmargin=\topskip,} 
\mdtheorem[style=theoremstyle]{definition}{Definition}

% concept environment
\mdfdefinestyle{conceptstyle}{% 
    linecolor=black,linewidth=1pt,% 
    frametitlerule=true,% 
    frametitlebackgroundcolor=lightskyblue, 
    innertopmargin=\topskip,} 
\mdtheorem[style=conceptstyle]{concept}{Concept}
\newcommand{\newterm}[1]{{\it #1}}

% Single space'd bib -
\setlength\bibsep{0pt}

\renewcommand{\rmdefault}{phv}\renewcommand{\sfdefault}{phv}
%\newboxedtheorem[boxcolor=black, background=gray!5,titlebackground=orange!20,titleboxcolor = black]{color_box_example}{Example}{test}

% Change the number format in the ref list -
\renewcommand{\bibnumfmt}[1]{#1.}

% Change Figure to Fig.
\renewcommand{\figurename}{Fig.}
\usepackage{enumitem}
\setlist{noitemsep} % or \setlist{noitemsep} to leave space around whole list

%Joycelyn Chan, Joshua Lequieu, Michael Paull, Chidanand Balaji, Ryan Tasseff
%Our derivation follows closely the earlier development of Fredrickson \citep{Fredrickson:1976fk}.

% Begin ...
\begin{document}

%\begin{titlepage}
{\par\centering\textbf{\Large CHEME 134 Module 1: American Options Contract Pricing}}
\vspace{0.2in}
{\par \centering \large{Jeffrey D. Varner}}
\vspace{0.05in}
{\par \centering \large{Smith School of Chemical and Biomolecular Engineering}}
{\par \centering \large{Cornell University, Ithaca NY 14853}}
% \vspace{0.1in}
% {\par \centering \small{Copyright \copyright\ Jeffrey Varner 2018. All Rights Reserved.}}\\

%\end{titlepage}
\date{}
\thispagestyle{empty}

\setcounter{page}{1}

\section*{Introduction}
American options contracts are financial instruments that give the holder the right, but not the obligation, 
to buy or sell an underlying asset at a specified price on or before a specified date. 
The price of the underlying asset is called the \newterm{spot price}, 
and the specified price where a transaction could occur is called the \newterm{strike price}. 
The date on which the option expires is called the \newterm{expiration date}. 
Finally, the price of the option itself is called the \newterm{premium}. American options contracts are similar to European options contracts, 
but with one key difference: American options contracts can be exercised at any time before the expiration date, 
while European options contracts can only be exercised on the expiration date. 
This difference makes American options contracts more complex to price than European options contracts
and leads to a rich set of dynamics in the contract pricing problem.

In this module, we will use the the \href{https://en.wikipedia.org/wiki/Binomial_options_pricing_model}{Cox-Ross-Rubinstein (CRR)} 
lattice model to simulate how different factors affect the price of American options contracts, e.g., the 
spot price of the underlying asset, the strike price, the expiration date, and the risk-free rate.
In addition, we will introduce the concepts of intrinsic and extrinsic value, which are key to understanding
the dynamics of American options contracts. 

\subsection*{The Cox-Ross-Rubinstein (CRR) Model}
The Cox-Ross-Rubinstein (CRR) model is a discrete-time model that simulates the price of an underlying asset over time
using a binomial lattice, while the price of the option contract is computed over the lattice using a backward induction algorithm \cite{COX1979229}.
During each time period, the price of the underlying asset can either go up by a factor $u$ or down by a factor $d$ 
with probabilities $p$ and $1-p$, respectively. The probability of an up move $p$ is a risk-neutral probability, 
which means that the expected return on the underlying asset is equal to the risk-free rate:
\begin{equation*}
p = \frac{\mathcal{D}_{1,0}(\bar{r}) - d}{u - d}
\end{equation*}
where $u$ and $d$ are the up and down factors, $\bar{r}$ denotes the risk-free rate, and $\mathcal{D}_{1,0}(\bar{r})$ 
is the continuous discount factor for one time step. 
The up factor $u$ in the CRR model is given by:
\begin{equation*}
    u = \exp(\sigma\cdot\sqrt{\Delta{t}})
\end{equation*}
where $\sigma$ is the \newterm{implied volatility} of the return of the underlying asset, and $\Delta{t}$ is the time step. 
Finally, in the CRR approach, the up and down factors are symmetric, i.e., $ud=1$.




\subsubsection*{What is the Implied Volatility?}
One of the initially confusing aspects of options pricing is the concept of \newterm{implied volatility}.
Implied volatility is the market's estimate of the future price volatility of the underlying asset.
It is \textit{implied} (as opposed to historical volatility) because it is estimated from the market price of the option contract itself.
Thus, the price action of the option contract reflects the market's expectation of the future volatility of the underlying asset.

The implied volatility represents one annualized standard deviation of the future spot price of the underlying asset over the next year.
We often express the implied volatility of an option contract as a percentage of the future spot price of the underlying asset, 
thus we typically write the implied volatility as a percentage, e.g., IV = 20\%. 
We use the implied volatility to compute the up factor $u$ in the CRR model, 
but we can also the IV to compute the market's expectation of the future share price of the underlying asset. 
If $\Delta{t}$ is a number of trading days in the future, then the market expects the share price of the underlying asset to be within the interval:
\begin{equation*}
S(\Delta{t}) = S_{0}\cdot\left(1\pm\frac{\text{IV}}{100}\cdot\sqrt{\frac{\Delta{t}}{252}}\right)
\end{equation*}
with a probability of 68\%, where $S_{0}$ is the current spot price of the underlying asset, 
and $\text{IV}$ is the implied volatility of the option contract expressed as a percentage.

\subsubsection*{Derivation of the $p$, $u$, and $d$ factors}
To better understand the CRR model, let's look at how the $p$, $u$, and $d$ factors are derived. The probability
of an up move $p$ is calculated by computing the expectation of the price of the underlying asset at the next time step.
One time step in the future, the price of the underlying asset can either go up by a factor $u$, which probability $p$ or down by a factor $d$
with probability $1-p$. Thus, the expectation of the share price of the underlying asset at the next time step $S_{j+1}$ is given by:
\begin{equation*}
	\mathbb{E}[S_{j+1}] = p\cdot{u}\cdot{S}_{j} + (1-p)\cdot{d}\cdot{S_{j}}
\end{equation*}
where $S_{j}$ is the price of the underlying asset at the current time step. 
The risk-neutral probability $p$ is then calculated by setting the expectation of the share price of the underlying asset at the next time step equal to: 
\begin{equation*}
	\mathbb{E}[S_{j+1}] = \mathcal{D}_{1,0}(\bar{r})\cdot{S}_{j}
\end{equation*}
where $\mathcal{D}_{1,0}(\bar{r})$ is the continuous discount factor for one time step, where we assume the discount rate is given by the risk-free rate $\bar{r}$.
Putting these two equations together, we can solve for the risk-neutral probability $p$:
\begin{equation*}
	p = \frac{\mathcal{D}_{1,0}(\bar{r}) - d}{u - d}
\end{equation*}
Thus, the probability of an up move $p$ is determined by the risk-free rate $\bar{r}$, the up factor $u$, and the down factor $d$, 
but not the implied volatility of the return of the underlying asset $\sigma$.  
Instead, the implied volatility of the return is critical in determining the up factor $u$.
The up factor $u$ is constructed so that the CRR model matches the variance of the return $\text{Var}(S_{j}/S_{j-1})=\sigma^{2}\Delta{t}$.
We know that the variance of the return is given by:
\begin{equation}
\mathbb{E}\left[(S_{j}/S_{j-1})^2\right] - \mathbb{E}\left[S_{j}/S_{j-1}\right]^{2} = \sigma^{2}\Delta{t}
\end{equation}




% \begin{equation*}
% 	p(u-1)^{2} + (1-p)(d-1)^{2} + \left[p(u-1)+(1-p)(d-1)\right]^{2} = \sigma^{2}\Delta{t}


% \begin{equation*}
% %     p(u-1)^{2} + (1-p)(d-1)^{2} + \left[p(u-1)+(1-p)(d-1)\right]^{2} = \sigma^{2}\Delta{t}
% \end{equation*}
% % which can be solved for $u$ (assuming $\Delta{t}^{2}$ terms and above are ignored):

\clearpage
\bibliography{References_v1}

\clearpage
\printindex

\end{document}
