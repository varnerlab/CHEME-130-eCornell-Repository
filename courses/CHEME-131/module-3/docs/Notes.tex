\documentclass[11pt]{article}

% load some asm stuff -
\usepackage{amssymb}
\usepackage{amsmath}
\usepackage{amsthm}
%\usepackage{palatino,lettrine}
\usepackage{fancyhdr}
\usepackage{epsfig}
\usepackage[round,comma,sort]{natbib}
\usepackage{simplemargins}
\usepackage{setspace}
\usepackage{wrapfig}
\usepackage{hyperref}
%\usepackage{boiboites}
\usepackage[margin=0pt,font=small,labelfont=bf]{caption}
\newcommand{\boldindex}[1]{\textbf{\hyperpage{#1}}}
\usepackage{makeidx}\makeindex
\bibliographystyle{plos2015}


\usepackage{algpseudocode}
\usepackage{algorithm}

% Set the size
%\textwidth = 6.75 in
%\textheight = 9.75 in
%\oddsidemargin = 0.0 in
%\evensidemargin = 0.0 in
%\topmargin = 0.01 in
%\headheight = 0.0 in
%\headsep = 0.25 in
%\parskip = 0.15in
% \doublespace
\setallmargins{1in}

\newtheorem{example}{Example}[section]
\newtheorem{thm}{Theorem}[section]
\newtheorem{property}{Property}[section]

\theoremstyle{definition}
\newtheorem{defn}[thm]{Definition}

\makeatletter
% \renewcommand\subsection{\@startsection
% 	{subsection}{2}{0mm}
% 	{-0.05in}
% 	{0.1\baselineskip}
% 	{\normalfont\normalsize\bfseries}}
\renewcommand\subsubsection{\@startsection
	{subsubsection}{2}{0mm}
	{-0.05in}
	{-0.5\baselineskip}
	{\normalfont\normalsize\itshape\bfseries}}
\renewcommand\paragraph{\@startsection
	{paragraph}{2}{0mm}
	{-0.05in}
	{-0.5\baselineskip}
	{\normalfont\normalsize\itshape}}
\makeatother
\linespread{1.1}

\fancypagestyle{proposal}{\fancyhf{}%
	\fancyhead[RO,LE]{\thepage}%
	\fancyhead[LO,RE]{CHEME 131 Module 3 STRIPS}%
	\renewcommand\headrulewidth{1pt}}
\pagestyle{proposal}

\usepackage{mdframed}
\definecolor{lgray}{rgb}{0.92,0.92,0.92}
\definecolor{antiquewhite}{rgb}{0.98,0.92,0.84}
\definecolor{lightskyblue}{rgb}{0.93,0.95,0.99}

% defn environment
\mdfdefinestyle{theoremstyle}{% 
    linecolor=black,linewidth=1pt,% 
    frametitlerule=true,% 
    frametitlebackgroundcolor=lgray, 
    innertopmargin=\topskip,} 
\mdtheorem[style=theoremstyle]{definition}{Definition}

% concept environment
\mdfdefinestyle{conceptstyle}{% 
    linecolor=black,linewidth=1pt,% 
    frametitlerule=true,% 
    frametitlebackgroundcolor=lightskyblue, 
    innertopmargin=\topskip,} 
\mdtheorem[style=conceptstyle]{concept}{Concept}
\newcommand{\newterm}[1]{{\it #1}}

% Single space'd bib -
\setlength\bibsep{0pt}

\renewcommand{\rmdefault}{phv}\renewcommand{\sfdefault}{phv}
%\newboxedtheorem[boxcolor=black, background=gray!5,titlebackground=orange!20,titleboxcolor = black]{color_box_example}{Example}{test}

% Change the number format in the ref list -
\renewcommand{\bibnumfmt}[1]{#1.}

% Change Figure to Fig.
\renewcommand{\figurename}{Fig.}
\usepackage{enumitem}
\setlist{noitemsep} % or \setlist{noitemsep} to leave space around whole list

%Joycelyn Chan, Joshua Lequieu, Michael Paull, Chidanand Balaji, Ryan Tasseff
%Our derivation follows closely the earlier development of Fredrickson \citep{Fredrickson:1976fk}.

% Begin ...
\begin{document}

%\begin{titlepage}
{\par\centering\textbf{\Large CHEME 131 Module 3: Registered Interest and Principal of Securities (STRIPS) Bonds}}
\vspace{0.2in}
{\par \centering \large{Jeffrey D. Varner}}
\vspace{0.05in}
{\par \centering \large{Smith School of Chemical and Biomolecular Engineering}}
{\par \centering \large{Cornell University, Ithaca NY 14853}}
% \vspace{0.1in}
% {\par \centering \small{Copyright \copyright\ Jeffrey Varner 2018. All Rights Reserved.}}\\

%\end{titlepage}
\date{}
\thispagestyle{empty}

\setcounter{page}{1}

\section*{Introduction}
\href{https://en.wikipedia.org/wiki/United_States_Treasury_security#STRIPS}{Registered Interest and Principal of Securities (STRIPS) bonds} are a 
unique type of fixed-income investment instrument that provides investors with an alternative way to access the coupon payments of 
Treasury securities. STRIPS bonds are created by separating a Treasury securities coupon and principal components and trading them as individual 
zero-coupon securities. This process allows investors to purchase and trade the coupon or principal components separately, 
providing greater flexibility in managing their investment portfolios.

\begin{figure}[h]
    \centering
    \includegraphics[width=0.85\textwidth]{./figs/Fig-STRIPS-Schematic.pdf}
    \caption{Schematic of a Registered Interest and Principal of Securities (STRIPS) bond generated from a 5-year Treasury note. The coupon and principle payments from the coupon-based 
	note are stripped fron the original instrument and sold as seperate marketable securities.}\label{fig:strips-bond-schematic}
\end{figure}

For example, a 5-year Treasury note with annual coupon payments of $C$ USD and a face (par) value of $V_{P}$ (USD)
can be stripped into six separate zero-coupon securities, i.e., five zero-coupon bonds, each with face values of $C$ 
and maturity of $T$= 1,2,3,4 and 5 years, and a six security with face  (par) value of $V_{P}$ USD with a duration of $T$ = 5 years (Fig. \ref{fig:strips-bond-schematic}). 
In the general case, a treasury note or bond with $N=\lambda{T}$ coupon payments, where $T$ denotes the maturity in years, and $\lambda$ represents 
the number of coupon payments per year, can be stripped into $N+1$ separate zero-coupon securities.
Beyond thier immediate value as investment tools, STRIPS are interesting as they provide look at the 
\href{https://www.federalreserve.gov/data/yield-curve-models.htm}{term structure of interest rates}, i.e., the relationship between the remaining time-to-maturity of debt securities 
and the yield on those securities.

In this module, we'll explore the mathematics of STRIPS bonds, and how they can be used to understand the term structure of interest rates, i.e., how we can use STRIPS to compute the short-rates, and the yield curve.

\section*{Spot, Short and Forward Rates}
A spot rate is the annual interest rate which is to be used for discounting the cash flows which occur at that date, i.e., now. Alternatively, the spot rate is the rate of return on a zero coupon bond that is purchased today and matures at some future date. Suppose a bond with a face value of $V_{P}$ (future USD) and a maturity of $T$ years is purchased today for $V_{B}$ (USD). The spot rate, $\bar{r}$, is the interest rate that makes the future face value of the bond equal to the purchase price, i.e.,
\begin{equation}
V_{B}\cdot(1+\bar{r})^{T} = V_{P}
\end{equation}
where we have assume one compounding event per year. Given the spot rate, we can compute the price of a bond with a face value of $V_{P}$ (USD) and a maturity of $T$ years or, alternatively, we can compute the spot rate $\bar{r}$ from the market price of a bond $V_{B}$:
\begin{equation}
\bar{r} = \left(\frac{V_{P}}{V_{B}}\right)^{\frac{1}{T}}-1
\end{equation}
Thus, the spot rate $\bar{r}$ is constant for the entire life of the bond which is a source of risk for the bondholder.
Why? Because we know that interest rates change in time. Thus, let's take a more nuianced view of the spot rate as a function of time which. This context gives us the short rate.

The short rate is the interest rate that occurs between two consecutive time periods, i.e., the interest rate that occurs between time $t$ and time $t+1$. We denote the short rate between time period $t$ and $t+1$ as $r_{t+1,t}$. The short rate is a random variable. which flucates in time. We've seen the short rate before in another guise, i.e., in the multiperiod discrete discountr factor, $D_{t,0}$, which is the discount factor between time $t = 0$ and $t = t$ when we assume the 
interest rate, i.e., the short rate, changes in each period of time. Of course, the short rates are related to the spot rate, $\bar{r}$, by the following relationship:
\begin{equation}
\prod_{j=0}^{t-1}\left(1+r_{j+1,j}\right) = \left(1+\bar{r}\right)^{t}
\end{equation}
where the product on the right-hand side is the discount factor between time $t = 0$ and $t = t$.
This must be true otherwise we could make money by borrowing at the short rate and investing at the spot rate.

Finally, the forward rate is the expectation of the short rate between two time periods, i.e., the expected interest rate that occurs between time $t$ and time $t+1$ given the information available at time $t$. 
The forward rate between period $j-1\rightarrow{j}$ is denoted by $f_{t+1,t}$, and is related to the short rate by the relationship:
\begin{equation}
	f_{t+1,t} = \mathbb{E}_{t}\left[r_{t+1,t}\right]
\end{equation}

\subsection*{Estimating the Short Rates from STRIPS}
For the moment, let's assume that we have a set of STRIPS zero-coupon bonds with maturities $T_{1},T_{2},\ldots,T_{N}$ years, 
and have the prices of these STRIPS, denoted as $V_{B,i}$.
We can estimate the short rates, which represent the market rate of interest between periods $j-1\rightarrow{j}$, denoted as $r_{j+1,j}$, by analyzing the prices of the various STRIPS zero coupon products based on their maturity. Using a discrete discounting model, the short rates are calculated according to the following relationship:
\begin{eqnarray}
\frac{V_{P,1}}{V_{B,1}} & = & \left(1+r_{1,0}\right) \\
\frac{V_{P,2}}{V_{B,2}} & = & \left(1+r_{2,1}\right)\cdot\left(1+r_{1,0}\right) \\
\vdots & = & \vdots \\
\frac{V_{P,N}}{V_{B,N}} & = & \prod_{i=1}^{N}\left(1+r_{i,i-1}\right)
\end{eqnarray}
where $V_{P,i}$ and $V_{B,i}$ denote the face (par) value and price of the $i^{th}$ zero-coupon bond (both of which are known). Thus, we can solve for $r_{1,0}$, then insert that into the following expression to solve for $r_{2,1}$, and so on. Systematically, we can solve for the log-transformed short rates as a system of linear algebraic equations (LAEs) of the from:
\begin{equation}
\mathbf{A}\mathbf{x} = \mathbf{b}
\end{equation}
where $x_{i} = \log\left(1+r_{i,i-1}\right)$, $b_{i} = \log\left(V_{P,i}/V_{B,i}\right)$ and $\mathbf{A}$ is a lower-triangular matrix of 1's. We solve for the log-transformed short rates using back/forward substitution,
and then transform these back to linear coordinates for each period: $r_{i,i-1} = 10^{x_{i}} - 1$.

\section*{STRIPS pricing}
STRIPS are sold at a discount compared to their face value. However, it's important to understand that the secondary seller, i.e., the brokerage splitting the original note or bond, determines the discount in the secondary treasury market by setting the price of the zero-coupon instrument. Let's imagine we have a set of STRIPS zero-coupon bonds with maturities $T_{1},T_{2},\ldots,T_{N}$ years. We can use the prices of these STRIPS to estimate the short rates, $r_{j+1,j}$, between time periods $j\rightarrow{j+1}$.
Alternatively, we can use the short rates to price the STRIPS products. To better understand this, let's propose two hypothetical pricing schemes that a brokerage might employ:
\begin{enumerate}
\item{\textbf{Scheme 1 constant yield}: In this approach, the brokerage prices the zero-coupon instruments to have a constant yield $\bar{r}$. This can be achieved by setting the price as an descalating fraction of the par value $V_{B} = \left(\alpha\right)^{T}\cdot{V}_{P}$, where $\alpha\leq{1}$, and $T$ represent the time to maturity of the generated zero coupon bond.}
\item{\textbf{Scheme 2 constant discount}: In this case, the brokerage prices the zero-coupon instruments to have a constant discount, i.e., the ratio of the price to the face value is constant across all instruments. In this case, $V_{B} = \left(\alpha\right)\cdot{V}_{P}$ is not a function of the time to maturity of the instrument, and $\alpha\leq{1}$.}
\end{enumerate}

\subsection*{Scheme 1: Constant Yield}
In this case, the price of the zero-coupon bond is given by $V_{B,i} = \left(\alpha\right)^{T_{i}}\cdot{V}_{P,i}$, where $\alpha\leq{1}$, and $T_{i}$ represent the time to maturity of the $ith$ generated zero coupon bond. However, this pricing expression can be rewitten as:
\begin{equation}
\left(\alpha\right)^{T_{i}} = \frac{V_{B,i}}{V_{P,i}} = \frac{1}{\left(1+\bar{r}\right)^{T_{i}}}
\end{equation}
where we can use to solve for the spot rate in terms of the descalating factor $\alpha$:
\begin{equation}
\bar{r} = \frac{1}{\alpha} - 1
\end{equation}
Thus, the spot rate is a function of the descalating factor $\alpha$, but under this pricing scheme, the spot rate is constant for all maturities. 

\subsection*{Scheme 2: Constant Discount}
In this case, the ratio of the price to the face value is constant across all instruments, i.e., $V_{B,i} = \left(\alpha\right)\cdot{V}_{P,i}$, where $\alpha\leq{1}$. However, this pricing expression can be rewitten as:
\begin{equation}
\alpha = \frac{V_{B,i}}{V_{P,i}} = \frac{1}{\left(1+\bar{r}\right)^{T_{i}}}
\end{equation}
where we can use to solve for the spot rate in terms of the descalating factor $\alpha$:
\begin{equation}
\bar{r} = \left(\frac{1}{\alpha}\right)^{1/T_{i}} - 1
\end{equation}
In this case, the spot rate is a function of the descalating factor $\alpha$, but under this pricing scheme, the spot rate is not constant for all maturities, i.e., it decreases as the time to maturity increases.

\section*{Summary}
In this module, we've explored the mathematics of STRIPS bonds, and how they can be used to understand the term structure of interest rates, i.e., how we can use STRIPS to compute the short-rates, and the yield curve. We've also explored two hypothetical pricing schemes that a brokerage might employ to price the zero-coupon instruments, and how these pricing schemes affect the spot rate. 

\end{document}
