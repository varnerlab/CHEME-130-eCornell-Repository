\documentclass[11pt]{article}

% load some asm stuff -
\usepackage{amssymb}
\usepackage{amsmath}
\usepackage{amsthm}
%\usepackage{palatino,lettrine}
\usepackage{fancyhdr}
\usepackage{epsfig}
\usepackage[square,sort,comma,numbers]{natbib}
\usepackage{simplemargins}
\usepackage{setspace}
\usepackage{wrapfig}
\usepackage{hyperref}
%\usepackage{boiboites}
\usepackage[margin=0pt,font=small,labelfont=bf]{caption}
\newcommand{\boldindex}[1]{\textbf{\hyperpage{#1}}}
\usepackage{makeidx}\makeindex
\bibliographystyle{plos2015}

\usepackage{algpseudocode}
\usepackage{algorithm}


% Set the size
%\textwidth = 6.75 in
%\textheight = 9.75 in
%\oddsidemargin = 0.0 in
%\evensidemargin = 0.0 in
%\topmargin = 0.01 in
%\headheight = 0.0 in
%\headsep = 0.25 in
%\parskip = 0.15in
% \doublespace
\setallmargins{1in}

\newtheorem{example}{Example}[section]
\newtheorem{thm}{Theorem}[section]
\newtheorem{property}{Property}[section]

\theoremstyle{definition}
\newtheorem{defn}[thm]{Definition}

\makeatletter
% \renewcommand\subsection{\@startsection
% 	{subsection}{2}{0mm}
% 	{-0.05in}
% 	{0.05\baselineskip}
% 	{\normalfont\normalsize\bfseries}}
\renewcommand\subsubsection{\@startsection
	{subsubsection}{2}{0mm}
	{-0.05in}
	{-0.5\baselineskip}
	{\normalfont\normalsize\itshape\bfseries}}
\renewcommand\paragraph{\@startsection
	{paragraph}{2}{0mm}
	{-0.05in}
	{-0.5\baselineskip}
	{\normalfont\normalsize\itshape}}
\makeatother
\linespread{1.1}

\fancypagestyle{proposal}{\fancyhf{}%
	\fancyhead[RO,LE]{\thepage}%
	\fancyhead[LO,RE]{CHEME 133 Module 4 Composite Contracts}%
	\renewcommand\headrulewidth{1pt}}
\pagestyle{proposal}

\usepackage{mdframed}
\definecolor{lgray}{rgb}{0.92,0.92,0.92}
\definecolor{antiquewhite}{rgb}{0.98,0.92,0.84}
\definecolor{lightskyblue}{rgb}{0.93,0.95,0.99}

% defn environment
\mdfdefinestyle{theoremstyle}{% 
    linecolor=black,linewidth=1pt,% 
    frametitlerule=true,% 
    frametitlebackgroundcolor=lgray, 
    innertopmargin=\topskip,} 
\mdtheorem[style=theoremstyle]{definition}{Definition}

% concept environment
\mdfdefinestyle{conceptstyle}{% 
    linecolor=black,linewidth=1pt,% 
    frametitlerule=true,% 
    frametitlebackgroundcolor=lightskyblue, 
    innertopmargin=\topskip,} 
\mdtheorem[style=conceptstyle]{concept}{Concept}
\newcommand{\newterm}[1]{{\it #1}}

% Single space'd bib -
\setlength\bibsep{0pt}

\renewcommand{\rmdefault}{phv}\renewcommand{\sfdefault}{phv}
%\newboxedtheorem[boxcolor=black, background=gray!5,titlebackground=orange!20,titleboxcolor = black]{color_box_example}{Example}{test}

% Change the number format in the ref list -
\renewcommand{\bibnumfmt}[1]{#1.}

% Change Figure to Fig.
\renewcommand{\figurename}{Fig.}
\usepackage{enumitem}
\setlist{noitemsep} % or \setlist{noitemsep} to leave space around whole list

%Joycelyn Chan, Joshua Lequieu, Michael Paull, Chidanand Balaji, Ryan Tasseff
%Our derivation follows closely the earlier development of Fredrickson \citep{Fredrickson:1976fk}.

% Begin ...
\begin{document}

%\begin{titlepage}
{\par\centering\textbf{\Large CHEME 133 Module 4: Analysis of American-Style Composite Options Contracts at Expiration}}
\vspace{0.2in}
{\par \centering \large{Jeffrey D. Varner}}
\vspace{0.05in}
{\par \centering \large{Smith School of Chemical and Biomolecular Engineering}}
{\par \centering \large{Cornell University, Ithaca NY 14853}}
% \vspace{0.1in}
% {\par \centering \small{Copyright \copyright\ Jeffrey Varner 2018. All Rights Reserved.}}\\

%\end{titlepage}
\date{}
\thispagestyle{empty}

\setcounter{page}{1}

\section*{Introduction}
Composite options contracts are financial instruments that are composed of two or more individual options contracts. 
The payoff and profite of a composite contract at expiration is the sum of the values of the individual contracts. 
The advantage of constructing composite contracts is that they can be used to construct complex payoffs and profits startegies 
from simple contract components. In this module, we will analyze composite contracts at expiration, 
where the composite contract is composed of two or more American-style options.

\subsection*{General Formulation}
Call and put contracts can be combined to develop composite contract structures with interesting payoff diagrams. 
Let $\mathcal{C}$ be a composite contract with $d$ legs (individual contracts) where each leg is written
with respect to the same underlying asset \texttt{XYZ} and same expiration date. 
Then, the payoff of the composite contract $\hat{V}(S(T),K_{1},\dots,K_{d})$ at time $T$ (expiration) is given by:
\begin{equation}
\hat{V}(S(T),K_{1},\dots,K_{d}) = \sum_{i\in\mathcal{C}}\theta_{i}\cdot{n_{i}}\cdot{V_{i}(S(T),K_{i}})
\end{equation}
where $K_{i}$ denotes the strike price of contract $i$, $\theta_{i}$ denotes the contract orientation $i$: $\theta_{i}=-1$ if contract $i$ is short (sold), 
otherwise $\theta_{i}=1$, and the quantity $n_{i}$ denotes the copy number of contract $i$.
The profit of the composite contract $\hat{P}$ at time $T$ (expiration) is given by:
\begin{equation}
\hat{P}(S(T),K_{1},\dots,K_{d}) = \sum_{i\in\mathcal{C}}\theta_{i}\cdot{n}_{i}\cdot{P}_{i}(S(T),K_{i})
\end{equation}
where $P_{i}(S(T),K_{i})$ denotes the profit of contract $i$. 
The profit of a contract is the payoff minus the initial cost of the contract, i.e., the premium paid to purchase the contract:
\begin{equation}
P_{i}(S(T),K_{i}) = V_{i}(S(T),K_{i}) - \theta_{i}\cdot{\mathcal{P}}_{i}
\end{equation}
where $\mathcal{P}_{i}$ denotes the premium paid to purchase contract $i$.

\section*{Directional Composite Contracts}
Directional composite contracts make a directional assumption about the price movement of the underlying asset, and can be opened for a credit or a debit.
A common directional composite contract is a \textit{spread}.

\subsection*{Credit Spreads}
A credit spread is a composite contract that is opened for a net credit.

\section*{Neutral Composite Contracts}
Neutral composite contracts make no directional assumption about the price movement of the underlying asset, and can be opened for a credit or a debit.
Two common directional composite contracts are the \textit{straddle} and the \textit{strangle}.	

\section*{Summary}
Fill me in.

\bibliography{References_v1}

\clearpage
\printindex

\end{document}
