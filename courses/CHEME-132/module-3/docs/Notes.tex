\documentclass[11pt]{article}

% load some asm stuff -
\usepackage{amssymb}
\usepackage{amsmath}
\usepackage{amsthm}
%\usepackage{palatino,lettrine}
\usepackage{fancyhdr}
\usepackage{epsfig}
\usepackage[square,sort,comma,numbers]{natbib}
\usepackage{simplemargins}
\usepackage{setspace}
\usepackage{wrapfig}
\usepackage{hyperref}
%\usepackage{boiboites}
\usepackage[margin=0pt,font=small,labelfont=bf]{caption}
\newcommand{\boldindex}[1]{\textbf{\hyperpage{#1}}}
\usepackage{makeidx}\makeindex
\bibliographystyle{plos2015}

\usepackage{algpseudocode}
\usepackage{algorithm}


% Set the size
%\textwidth = 6.75 in
%\textheight = 9.75 in
%\oddsidemargin = 0.0 in
%\evensidemargin = 0.0 in
%\topmargin = 0.01 in
%\headheight = 0.0 in
%\headsep = 0.25 in
%\parskip = 0.15in
% \doublespace
\setallmargins{1in}

\newtheorem{example}{Example}[section]
\newtheorem{thm}{Theorem}[section]
\newtheorem{property}{Property}[section]

\theoremstyle{definition}
\newtheorem{defn}[thm]{Definition}

\makeatletter
% \renewcommand\subsection{\@startsection
% 	{subsection}{2}{0mm}
% 	{-0.05in}
% 	{0.05\baselineskip}
% 	{\normalfont\normalsize\bfseries}}
\renewcommand\subsubsection{\@startsection
	{subsubsection}{2}{0mm}
	{-0.05in}
	{-0.5\baselineskip}
	{\normalfont\normalsize\itshape\bfseries}}
\renewcommand\paragraph{\@startsection
	{paragraph}{2}{0mm}
	{-0.05in}
	{-0.5\baselineskip}
	{\normalfont\normalsize\itshape}}
\makeatother
\linespread{1.1}

\fancypagestyle{proposal}{\fancyhf{}%
	\fancyhead[RO,LE]{\thepage}%
	\fancyhead[LO,RE]{CHEME 132 Module 3 Multi Asset Geometric Brownian Motion}%
	\renewcommand\headrulewidth{1pt}}
\pagestyle{proposal}

\usepackage{mdframed}
\definecolor{lgray}{rgb}{0.92,0.92,0.92}
\definecolor{antiquewhite}{rgb}{0.98,0.92,0.84}
\definecolor{lightskyblue}{rgb}{0.93,0.95,0.99}

% defn environment
\mdfdefinestyle{theoremstyle}{% 
    linecolor=black,linewidth=1pt,% 
    frametitlerule=true,% 
    frametitlebackgroundcolor=lgray, 
    innertopmargin=\topskip,} 
\mdtheorem[style=theoremstyle]{definition}{Definition}

% concept environment
\mdfdefinestyle{conceptstyle}{% 
    linecolor=black,linewidth=1pt,% 
    frametitlerule=true,% 
    frametitlebackgroundcolor=lightskyblue, 
    innertopmargin=\topskip,} 
\mdtheorem[style=conceptstyle]{concept}{Concept}
\newcommand{\newterm}[1]{{\it #1}}

% Single space'd bib -
\setlength\bibsep{0pt}

\renewcommand{\rmdefault}{phv}\renewcommand{\sfdefault}{phv}
%\newboxedtheorem[boxcolor=black, background=gray!5,titlebackground=orange!20,titleboxcolor = black]{color_box_example}{Example}{test}

% Change the number format in the ref list -
\renewcommand{\bibnumfmt}[1]{#1.}

% Change Figure to Fig.
\renewcommand{\figurename}{Fig.}
\usepackage{enumitem}
\setlist{noitemsep} % or \setlist{noitemsep} to leave space around whole list

%Joycelyn Chan, Joshua Lequieu, Michael Paull, Chidanand Balaji, Ryan Tasseff
%Our derivation follows closely the earlier development of Fredrickson \citep{Fredrickson:1976fk}.

% Begin ...
\begin{document}

%\begin{titlepage}
{\par\centering\textbf{\Large CHEME 132 Module 3: Multiple Asset Geometric Brownian Motion}}
\vspace{0.2in}
{\par \centering \large{Jeffrey D. Varner}}
\vspace{0.05in}
{\par \centering \large{Smith School of Chemical and Biomolecular Engineering}}
{\par \centering \large{Cornell University, Ithaca NY 14853}}
% \vspace{0.1in}
% {\par \centering \small{Copyright \copyright\ Jeffrey Varner 2018. All Rights Reserved.}}\\

%\end{titlepage}
\date{}
\thispagestyle{empty}

\setcounter{page}{1}

\section*{Introduction}
\href{https://en.wikipedia.org/wiki/Geometric_Brownian_motion}{Geometric Brownian motion (GBM)} is a continuous-time stochastic model in which the random variable $S(t)$, 
e.g., the share price of a firm. 
Geometric Brownian motion was popularized as a financial model by Samuelson in the 1950s and 1960s \cite{Merton2006}, 
but is arguably most commonly associated with the Black–Scholes options pricing model, which we will describe later 
\cite{BlackScholes1973}. Previously, we considered the single asset GBM model, 
where the share price of a firm is modeled as a deterministic drift term that is corrupted by a Wiener noise process. 
Now, we will consider the multi-asset GBM model, where the share price of a firm is modeled as a deterministic drift term that is corrupted by a Wiener noise process, and the share price of a firm is correlated with the share price of other firms
through the noise process.

Consider an asset portfolio $\mathcal{P}$ with a return covariance matrix $\Sigma$ and drift vector $\mu$.
The multi-dimensional geometric Brownian motion model describing the share price $S_{i}(t)$ for asset $i\in\mathcal{P}$ is given by: 
\begin{equation*}
\frac{dS_{i}\left(t\right)}{S_{i}(t)} = \mu_{i}\,{dt}+\sum_{j=1}^{\mathcal{P}}a_{ij}\cdot{dW_{j}(t)}\qquad\text{for}\quad{i=1,2,\dots,\mathcal{P}}
\end{equation*}
where $a_{ij}\in\mathbf{A}$ and $\mathbf{A}\mathbf{A}^{\top} = \Sigma$, and $\mu_{i}$ denotes the drift parameter for asset $i$. 
The multi-dimensional GBM model has the analytical solution:
\begin{equation*}
S_{i}(t_{k}) = S_{i}(t_{k-1})\cdot\exp\Biggl[\left(\mu_{i}-\frac{\sigma_{i}^{2}}{2}\right)\Delta{t} + \sqrt{\Delta{t}}\cdot\sum_{j\in\mathcal{P}}a_{ij}\cdot{Z_{j}(0,1)}\Biggr]\quad{i\in\mathcal{P}}
\end{equation*}
where $S_{i}(t_{k-1})$ is the share price at time $t_{k-1}$ for asset $i\in\mathcal{P}$,  $\Delta{t} = t_{k} - t_{k-1}$ is the time difference (step-size) for each time step (fixed), 
and $Z_{j}(0,1)$ is a standard normal random variable. 

\section*{Estimating the covariance matrix $\Sigma$}
The covariance between the logarithmic return $r_{\star}$ on assets $i$ and $j$, denoted as $\text{cov}\left(r_{i},r_{j}\right)$, quantifies the relationship
between assets $i$ and $j$ in the portfolio $\mathcal{P}$. 
\begin{equation*}
    \Sigma_{ij} = \text{cov}\left(r_{i},r_{j}\right) = \sigma_{i}\sigma_{j}\rho_{ij}\qquad\text{for}\quad{i,j \in \mathcal{P}}
\end{equation*}
where $\sigma_{\star}$ denote the standard deviation of the return of asset $\star$, and $\rho_{ij}$ 
denotes the correlation between assets $i$ and $j$ in the portfolio $\mathcal{P}$. The correlation is given by:
\begin{equation*}
\rho_{ij} = \frac{\mathbb{E}(r_{i}-\mu_{i})\cdot\mathbb{E}(r_{j} - \mu_{j})}{\sigma_{i}\cdot\sigma_{j}}\qquad\text{for}\quad{i,j \in \mathcal{P}}
\end{equation*}
The diagonal elements of the covariance matrix $\Sigma$ are the variances, 
while the off-diagonal measure the relationship between the assets in the portfolio $\mathcal{P}$.

\section*{Summary}
Fill me in.

\bibliography{References_v1}

\clearpage
\printindex

\end{document}
