\documentclass[11pt]{article}

% load some asm stuff -
\usepackage{amssymb}
\usepackage{amsmath}
\usepackage{amsthm}
%\usepackage{palatino,lettrine}
\usepackage{fancyhdr}
\usepackage{epsfig}
\usepackage[square,sort,comma,numbers]{natbib}
\usepackage{simplemargins}
\usepackage{setspace}
\usepackage{wrapfig}
\usepackage{hyperref}
%\usepackage{boiboites}
\usepackage[margin=0pt,font=small,labelfont=bf]{caption}
\newcommand{\boldindex}[1]{\textbf{\hyperpage{#1}}}
\usepackage{makeidx}\makeindex
\bibliographystyle{plos2015}

\usepackage{algpseudocode}
\usepackage{algorithm}


% Set the size
%\textwidth = 6.75 in
%\textheight = 9.75 in
%\oddsidemargin = 0.0 in
%\evensidemargin = 0.0 in
%\topmargin = 0.01 in
%\headheight = 0.0 in
%\headsep = 0.25 in
%\parskip = 0.15in
% \doublespace
\setallmargins{1in}

\newtheorem{example}{Example}[section]
\newtheorem{thm}{Theorem}[section]
\newtheorem{property}{Property}[section]

\theoremstyle{definition}
\newtheorem{defn}[thm]{Definition}

\makeatletter
% \renewcommand\subsection{\@startsection
% 	{subsection}{2}{0mm}
% 	{-0.05in}
% 	{0.05\baselineskip}
% 	{\normalfont\normalsize\bfseries}}
\renewcommand\subsubsection{\@startsection
	{subsubsection}{2}{0mm}
	{-0.05in}
	{-0.5\baselineskip}
	{\normalfont\normalsize\itshape\bfseries}}
\renewcommand\paragraph{\@startsection
	{paragraph}{2}{0mm}
	{-0.05in}
	{-0.5\baselineskip}
	{\normalfont\normalsize\itshape}}
\makeatother
\linespread{1.1}

\fancypagestyle{proposal}{\fancyhf{}%
	\fancyhead[RO,LE]{\thepage}%
	\fancyhead[LO,RE]{CHEME 132 Module 3 Multi Asset Geometric Brownian Motion}%
	\renewcommand\headrulewidth{1pt}}
\pagestyle{proposal}

\usepackage{mdframed}
\definecolor{lgray}{rgb}{0.92,0.92,0.92}
\definecolor{antiquewhite}{rgb}{0.98,0.92,0.84}
\definecolor{lightskyblue}{rgb}{0.93,0.95,0.99}

% defn environment
\mdfdefinestyle{theoremstyle}{% 
    linecolor=black,linewidth=1pt,% 
    frametitlerule=true,% 
    frametitlebackgroundcolor=lgray, 
    innertopmargin=\topskip,} 
\mdtheorem[style=theoremstyle]{definition}{Definition}

% concept environment
\mdfdefinestyle{conceptstyle}{% 
    linecolor=black,linewidth=1pt,% 
    frametitlerule=true,% 
    frametitlebackgroundcolor=lightskyblue, 
    innertopmargin=\topskip,} 
\mdtheorem[style=conceptstyle]{concept}{Concept}
\newcommand{\newterm}[1]{{\it #1}}

% Single space'd bib -
\setlength\bibsep{0pt}

\renewcommand{\rmdefault}{phv}\renewcommand{\sfdefault}{phv}
%\newboxedtheorem[boxcolor=black, background=gray!5,titlebackground=orange!20,titleboxcolor = black]{color_box_example}{Example}{test}

% Change the number format in the ref list -
\renewcommand{\bibnumfmt}[1]{#1.}

% Change Figure to Fig.
\renewcommand{\figurename}{Fig.}
\usepackage{enumitem}
\setlist{noitemsep} % or \setlist{noitemsep} to leave space around whole list

%Joycelyn Chan, Joshua Lequieu, Michael Paull, Chidanand Balaji, Ryan Tasseff
%Our derivation follows closely the earlier development of Fredrickson \citep{Fredrickson:1976fk}.

% Begin ...
\begin{document}

%\begin{titlepage}
{\par\centering\textbf{\Large CHEME 132 Module 3: Multiple Asset Geometric Brownian Motion}}
\vspace{0.2in}
{\par \centering \large{Jeffrey D. Varner}}
\vspace{0.05in}
{\par \centering \large{Smith School of Chemical and Biomolecular Engineering}}
{\par \centering \large{Cornell University, Ithaca NY 14853}}
% \vspace{0.1in}
% {\par \centering \small{Copyright \copyright\ Jeffrey Varner 2018. All Rights Reserved.}}\\

%\end{titlepage}
\date{}
\thispagestyle{empty}

\setcounter{page}{1}

\section*{Introduction}
\href{https://en.wikipedia.org/wiki/Geometric_Brownian_motion}{Geometric Brownian motion (GBM)} is a continuous-time stochastic model in which the random variable $S(t)$, 
e.g., the share price of a firm, is described by a stochastic differential equation.
Geometric Brownian motion was popularized as a financial model by Samuelson in the 1950s and 1960s \cite{Merton2006}, 
but is arguably most commonly associated with the Black–Scholes options pricing model, which we will describe later 
\cite{BlackScholes1973}. Previously, we considered the single asset GBM model, 
where the share price of a firm was modeled as a deterministic drift term that is corrupted by a Wiener noise process. 
We showed that the share price of a firm follows a log-normal distribution, and derived the analytical solution for the share price of a firm at a future time point.
However, in practice, investors often hold portfolios of assets, and the share price of a firm is correlated with the share price of other firms.
Thus, let's consider the GBM for multiple simulataneous assets, where the share price of a firm is modeled as a deterministic drift term 
that is corrupted by a Wiener noise process (same as before), but now we consider how the noise is correlated with other firms in a portfolio.

Consider an asset portfolio $\mathcal{P}$, e.g., a collection of equities with a logarithmic return covariance matrix $\Sigma$ and drift vector $\mu$.
The multi-dimensional geometric Brownian motion model describing the share price $S_{i}(t)$ for asset $i\in\mathcal{P}$ at time $t$ is given by: 
\begin{equation*}
\frac{dS_{i}\left(t\right)}{S_{i}(t)} = \mu_{i}\,{dt}+\sum_{j=1}^{\mathcal{P}}a_{ij}\cdot{dW_{j}(t)}\qquad\text{for}\quad{i=1,2,\dots,\mathcal{P}}
\end{equation*}
where $a_{ij}\in\mathbf{A}$ and $\mathbf{A}\mathbf{A}^{\top} = \Sigma$ are noise coefficients describing the connection between firms $i$ and firms $j$ in the portfolio $\mathcal{P}$,
and $\mu_{i}$ denotes the drift parameter for asset $i$ in the portfolio $\mathcal{P}$.
The multi-dimensional GBM model has the analytical solution:
\begin{equation*}
S_{i}(t_{k}) = S_{i}(t_{k-1})\cdot\exp\Biggl[\left(\mu_{i}-\frac{\sigma_{i}^{2}}{2}\right)\Delta{t} + \sqrt{\Delta{t}}\cdot\sum_{j\in\mathcal{P}}a_{ij}\cdot{Z_{j}(0,1)}\Biggr]\quad{i\in\mathcal{P}}
\end{equation*}
where $S_{i}(t_{k-1})$ is the share price for firm $i\in\mathcal{P}$ at time $t_{k-1}$,  the term $\Delta{t} = t_{k} - t_{k-1}$ denotes the time difference (step-size) 
for each time step (fixed), and $Z_{j}(0,1)$ is a standard normal random variable for firm $j\in\mathcal{P}$.

\section*{Estimating the drift vector $\mu$ and covariance matrix $\Sigma$}
The drift vector $\mu$ and covariance matrix $\Sigma$ are key parameters in the multi-dimensional GBM model.
We can estimate the drift vector $\mu$, which will be a $\dim\mathcal{P}\times{1}$ vector, from historical data by computing the logarithmic excess growth rate of the assets in the portfolio $\mathcal{P}$, 
as we have shown previosly; we simply repeat the process for each asset in the portfolio $\mathcal{P}$.
On the other hand, the covariance matrix, a $\dim\mathcal{P}\times\dim\mathcal{P}$ symmetric matrix, describes the relationship between the logarithmic return series of firms $i$ and $j$.
The $(i,j)$ element of the covariance matrix $\sigma_{ij}\in\Sigma$ is given by:
\begin{equation*}
    \sigma_{ij} = \text{cov}\left(r_{i},r_{j}\right) = \sigma_{i}\sigma_{j}\rho_{ij}\qquad\text{for}\quad{i,j \in \mathcal{P}}
\end{equation*}
where $\sigma_{\star}$ denote the standard deviation of the logarithmic return of asset $\star$, i.e., the volatility parameter, and $\rho_{ij}$ 
denotes the correlation between the returns of asset $i$ and $j$ in the portfolio $\mathcal{P}$. The correlation is given by:
\begin{equation*}
\rho_{ij} = \frac{\mathbb{E}(r_{i}-\mu_{i})\cdot\mathbb{E}(r_{j} - \mu_{j})}{\sigma_{i}\cdot\sigma_{j}}\qquad\text{for}\quad{i,j \in \mathcal{P}}
\end{equation*}
where $\mathbb{E}(r_{i}-\mu_{i})$ is the expected value of the difference between the logarithmic return of asset $i$ and its drift parameter 
$\mu_{i}$, i.e., mean of the logorithmic return. The diagonal elements of the covariance matrix $\sigma_{ii}\in\Sigma$ are the variances, 
while the off-diagonal measure the relationship between assets $i$ and $j$ in the portfolio $\mathcal{P}$.

\begin{algorithm}[h]
    \caption{Logarithmic Excess Growth Rate}\label{algo-log-return-distributions-equity}
    \begin{algorithmic}[1]

        \Statex
        \Require data set $\mathcal{D}_{i} = \left\{S_{i,t}\right\}_{t=1}^{N}\in\mathcal{D}$ where $S_{i,t}$ denotes the price of stock $i$ at time $t$, all stocks have the same time horizon $N\gg{2}$, 
		and $\mathcal{D}$ denotes the data set of all stocks.
        \Require The time interval $\Delta{t}$ between $t$ and $t-1$ (units: years), and a list of stocks $\mathcal{L} = \left\{i\right\}_{i=1}^{M}$ where $M = \dim\mathcal{L}$.
        \Require The risk-free rate $r_{f}$ (units: inverse years).
     
        \Statex
		\Procedure{log growth rate}{$\mathcal{D}$, $\mathcal{L}$, $\Delta{t}$, $r_{f}$}
		\State{$N\leftarrow\text{length}(\mathcal{D})$}\Comment{Number of trading days for each stock $i\in\mathcal{L}$}
        \For{$i\in\mathcal{L}$}
			\State{$\mathcal{D}_{i} \gets \mathcal{D}[i]$}\Comment{Select the data for stock $i$ from the dataset collection $\mathcal{D}$}
            \For{$t=2\rightarrow{N}$}
                \State{$S_{i,t-1} \gets \mathcal{D}_{i}[t-1]$}\Comment{Select the price of stock $i$ at time $t-1$}
                \State{$S_{i,t} \gets \mathcal{D}_{i}[t]$}\Comment{Select the price of stock $i$ at time $t$}
                \State{$\mu^{(i)}_{t,t-1} \gets \left(1/\Delta{t}\right)\cdot\ln\left(S_{i,t}/{S_{i,t-1}}\right) - r_{f}$}\Comment{Set $r_{f} = 0$ for regular growth rate}
            \EndFor
        \EndFor
        \Statex
        \Return{$\mu^{(1)},\dots,\mu^{(\dim\mathcal{L})}$}\Comment{Return the logarithmic growth rate array for each stock $i\in\mathcal{L}$}
		\EndProcedure
    \end{algorithmic}
\end{algorithm}

\section*{Estimating the noise coefficients $a_{ij}$}
The noise coefficients $a_{ij}$ modify the noise terms in the multi-dimensional GBM model, 
and describe the connection between firms $i$ and firms $j$ in the portfolio $\mathcal{P}$. 
The noise coefficients are given by the Cholesky decomposition of the covariance matrix $\Sigma$:
\begin{equation*}
\Sigma = \mathbf{A}\mathbf{A}^{\top}
\end{equation*}
where $\mathbf{A}$ is a lower triangular matrix. 
The Cholesky decomposition is a matrix factorization that decomposes the covariance matrix $\Sigma$ 
into the product of a lower triangular matrix $\mathbf{A}$ and its complex conjugate transpose $\mathbf{A}^{\top}$;
given that $\Sigma$ is a positive-definite matrix, the Cholesky decomposition is unique.

\section*{Multiasset GBM simulation algorithm}
Now that we have estimated the drift vector $\mu$, the covariance matrix $\Sigma$, and the noise coefficients $a_{ij}$, we can
simulate the multi-dimensional GBM model to predict the share price of a firm at a future time point. 
Much like the single asset GBM model, we can simulate the multi-dimensional GBM model using the analytical solution, 
except in this case we will simulate the share price of each firm in the portfolio $\mathcal{P}$ at each time point.
Thus, we'll need a vector of initial share prices $\mathbf{S}(t_{0})$, the drift vector $\mu$, the covariance matrix $\Sigma$,
along with user-defined time points $t_{0},t_{1},\dots,t_{N}$, and the number of samples to simulate $N_{\text{samples}}$.
Given this data, we can use the following algorithm to simulate the multi-dimensional GBM model (\ref{algo-multi-asset-gbm-simulation}).


\begin{algorithm}[h]
    \caption{Logarithmic Excess Growth Rate}\label{algo-multi-asset-gbm-simulation}
    \begin{algorithmic}[1]
		\Require The initial share price vector $\mathbf{S}(t_{0}) = \left\{S_{i}(t_{0})\right\}_{i=1}^{\dim\mathcal{P}}$ for each firm $i\in\mathcal{P}$.
		\Require The drift vector $\mu = \left\{\mu_{i}\right\}_{i=1}^{\dim\mathcal{P}}$ for each firm $i\in\mathcal{P}$.
		\Require The covariance matrix $\Sigma$ for the portfolio $\mathcal{P}$.
		\Require The time step $\Delta{t}$ (units: years), the initial time $t_{\circ}$, the final time $t_{f}$, and the number of samples $N_{\text{samples}}$.
		\Require A Cholesky decomposition function $\texttt{cholesky}(\Sigma)$ that returns the Cholesky factor $\mathbf{A}$.

		\Statex
		\Procedure{Multiasset GBM}{$\mu$, $\Sigma$, $\mathbf{S}_{\circ}$, $(t_{\circ}, t_{f}, \Delta{t})$, $N_{\text{samples}}$}
		\State{$N_{a} \gets \text{length}(\mu)$}\Comment{Number of assets in the portfolio $\mathcal{P}$}
		\State{$N_{t} \gets (t_{f} - t_{\circ})/\Delta{t}$}\Comment{Number of time steps}
		\State{$X \gets \texttt{zeros}(N_{\text{samples},N_{t},N_{a}+1})$}\Comment{Pre-allocate the share price array}
		\State{$\mathbf{A} \gets \texttt{cholesky}(\Sigma)$}\Comment{Cholesky decomposition of the covariance matrix $\Sigma$}

		\EndProcedure
	\end{algorithmic}
\end{algorithm}

\section*{Summary}
Fill me in.

\bibliography{References_v1}

\clearpage
\printindex

\end{document}
