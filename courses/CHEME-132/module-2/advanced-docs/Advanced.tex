\documentclass[11pt]{article}

% load some asm stuff -
\usepackage{amssymb}
\usepackage{amsmath}
\usepackage{amsthm}
%\usepackage{palatino,lettrine}
\usepackage{fancyhdr}
\usepackage{epsfig}
\usepackage[square,sort,comma,numbers]{natbib}
\usepackage{simplemargins}
\usepackage{setspace}
\usepackage{wrapfig}
\usepackage{hyperref}
%\usepackage{boiboites}
\usepackage[margin=0pt,font=small,labelfont=bf]{caption}
\newcommand{\boldindex}[1]{\textbf{\hyperpage{#1}}}
\usepackage{makeidx}\makeindex
\bibliographystyle{plos2015}

\usepackage{algpseudocode}
\usepackage{algorithm}


% Set the size
%\textwidth = 6.75 in
%\textheight = 9.75 in
%\oddsidemargin = 0.0 in
%\evensidemargin = 0.0 in
%\topmargin = 0.01 in
%\headheight = 0.0 in
%\headsep = 0.25 in
%\parskip = 0.15in
% \doublespace
\setallmargins{1in}

\newtheorem{example}{Example}[section]
\newtheorem{thm}{Theorem}[section]
\newtheorem{property}{Property}[section]

\theoremstyle{definition}
\newtheorem{defn}[thm]{Definition}

\makeatletter
% \renewcommand\subsection{\@startsection
% 	{subsection}{2}{0mm}
% 	{-0.05in}
% 	{0.05\baselineskip}
% 	{\normalfont\normalsize\bfseries}}
\renewcommand\subsubsection{\@startsection
	{subsubsection}{2}{0mm}
	{-0.05in}
	{-0.5\baselineskip}
	{\normalfont\normalsize\itshape\bfseries}}
\renewcommand\paragraph{\@startsection
	{paragraph}{2}{0mm}
	{-0.05in}
	{-0.5\baselineskip}
	{\normalfont\normalsize\itshape}}
\makeatother
\linespread{1.1}

\fancypagestyle{proposal}{\fancyhf{}%
	\fancyhead[RO,LE]{\thepage}%
	\fancyhead[LO,RE]{CHEME 132 Module 2 Advanced Single Asset Geometric Brownian Motion}%
	\renewcommand\headrulewidth{1pt}}
\pagestyle{proposal}

\usepackage{mdframed}
\definecolor{lgray}{rgb}{0.92,0.92,0.92}
\definecolor{antiquewhite}{rgb}{0.98,0.92,0.84}
\definecolor{lightskyblue}{rgb}{0.93,0.95,0.99}

% defn environment
\mdfdefinestyle{theoremstyle}{% 
    linecolor=black,linewidth=1pt,% 
    frametitlerule=true,% 
    frametitlebackgroundcolor=lgray, 
    innertopmargin=\topskip,} 
\mdtheorem[style=theoremstyle]{definition}{Definition}

% concept environment
\mdfdefinestyle{conceptstyle}{% 
    linecolor=black,linewidth=1pt,% 
    frametitlerule=true,% 
    frametitlebackgroundcolor=lightskyblue, 
    innertopmargin=\topskip,} 
\mdtheorem[style=conceptstyle]{concept}{Concept}
\newcommand{\newterm}[1]{{\it #1}}

% Single space'd bib -
\setlength\bibsep{0pt}

\renewcommand{\rmdefault}{phv}\renewcommand{\sfdefault}{phv}
%\newboxedtheorem[boxcolor=black, background=gray!5,titlebackground=orange!20,titleboxcolor = black]{color_box_example}{Example}{test}

% Change the number format in the ref list -
\renewcommand{\bibnumfmt}[1]{#1.}

% Change Figure to Fig.
\renewcommand{\figurename}{Fig.}
\usepackage{enumitem}
\setlist{noitemsep} % or \setlist{noitemsep} to leave space around whole list

%Joycelyn Chan, Joshua Lequieu, Michael Paull, Chidanand Balaji, Ryan Tasseff
%Our derivation follows closely the earlier development of Fredrickson \citep{Fredrickson:1976fk}.

% Begin ...
\begin{document}

%\begin{titlepage}
{\par\centering\textbf{\Large CHEME 132 Module 2 Advanced: Weighted Returns, Time Depending Parameters and Dividends for Single Asset Geometric Brownian Motion}}
\vspace{0.2in}
{\par \centering \large{Jeffrey D. Varner}}
\vspace{0.05in}
{\par \centering \large{Smith School of Chemical and Biomolecular Engineering}}
{\par \centering \large{Cornell University, Ithaca NY 14853}}
% \vspace{0.1in}
% {\par \centering \small{Copyright \copyright\ Jeffrey Varner 2018. All Rights Reserved.}}\\

%\end{titlepage}
\date{}
\thispagestyle{empty}

\setcounter{page}{1}

\section*{Introduction}
\href{https://en.wikipedia.org/wiki/Geometric_Brownian_motion}{Geometric Brownian motion (GBM)} is a continuous-time stochastic model in which the random variable $S(t)$, 
e.g., the share price of a firm. 
Geometric Brownian motion was popularized as a financial model by Samuelson in the 1950s and 1960s \cite{Merton2006}, 
but is arguably most commonly associated with the Black–Scholes options pricing model, which we will describe later 
\cite{BlackScholes1973}. Geometric Brownian motion (GBM) assumes that the share price $S(t)$ of a firm can be modeled as a deterministic
drift term (which is proportional to the share price) that is corrupted by a \href{https://en.wikipedia.org/wiki/Wiener_process}{Wiener noise process}, also proportional to the share price:
\begin{equation}\label{eqn:GBM}
\frac{dS}{S} = {\mu}\,dt + \sigma\,{dW}
\end{equation}
The constant $\mu$ denotes a drift parameter, i.e., the growth rate of the share price return, $\sigma>0$ is a volatility parameter, i.e., 
the dispersion of the return, $dt$ denotes an infinitesimal time step, and $dW$ is the output of a 
\href{https://en.wikipedia.org/wiki/Wiener_process}{Wiener noise process}.  A \href{https://en.wikipedia.org/wiki/Wiener_process)}{Wiener Process} 
(also often referred to as a standard Brownian motion) is a real-valued continuous-time stochastic 
process named after \href{https://en.wikipedia.org/wiki/Norbert_Wiener}{Norbert Wiener} for the study of one-dimensional Brownian motion (Defn. \ref{defn-wiener-process}):
\begin{definition}[Wiener Process]\label{defn-wiener-process}
A Wiener process is a continuous one-dimensional stochastic process $\left\{W\left(t\right), 0\leq{t}\leq{T}\right\}$ with the following properties:
\begin{itemize}
\setlength\itemsep{0em}
\item{W$\left(0\right)$ = $0$ with probability $1$}
\item{The increments $\left\{W(t_{1}) - W(t_{o}),\dots, W(t_{k}) - W(t_{k-1})\right\}$ are independent for any $k$ and $0\leq{t_{o}}< t_{1} < \dots < t_{k} \leq{T}$}
\item{The increment W(t) - W(s) $\sim~N\left(0,t-s\right)$ for any $0\leq{s}< t \leq{T}$, where $N\left(0,t-s\right)$ denotes a normally distributed random variable with mean $0$ and variance $t - s$.}
\end{itemize}
\end{definition}
Equation \ref{eqn:GBM} is a continuous-time analog of the discrete-time binomial lattice model we developed previously. 
It has several exciting properties; for example, it has an analytical solution and analytical expressions 
for the expectation and variance of the share price. In this module, we will develop analytical solutions to Eqn. \ref{eqn:GBM}, 
and tools to estimate the parameters $\mu$ and $\sigma$ from historical data. We'll then use these tools to simulate the share 
price of firms and analyze properties of the return distributions in the example and project for this module. 

\section*{Weighted Expected Return}
Previously, we used the arithmetic mean of historical data to estimate the expected annualized growth rate $\mu$ for a particular firm.
However, one could argue that the arithmetic mean is not the best estimator of the expected growth rate in the current market environment.
For example, the arithmetic mean is sensitive to outliers and does not account for the time-varying nature of the growth rate 
or current macroeconomic conditions. To address these issues, we will develop a weighted return estimator for the expected growth rate $\mu$
that be be used to focus on more recent data, or to down-weight the impact of outliers. 

The weighted return estimator is a generalization of the arithmetic mean that assigns a weight to each data point in the historical data set.
Suppose we had a data set $\mathcal{D}$ that contained the close, or the volume-weighted-average (VWAP) price 
or some other price associated with a firm $i$ over $T$ periods, e.g., VWAP samples from periods $1\rightarrow{T}$.
Then, from the definition of discrete expectation, we know the expected excess return for firm $i$ is given by:

\begin{equation}\label{eqn:expected-return}
\mathbb{E}\left(R_{i}\right) = \sum_{t=1}^{T}p_{i}(t)\cdot{R_{i,t}}
\end{equation}
where $R_{i,t}$ is the excess return for firm $i$ at time $t$, 
and $p_{i}(t)$ denotes the probability of observing the excess return $R_{i,t}$ at time $t$ for firm $i$, 
where the probability $\sum_{t}p_{i}(t) = 1$. 
The probability terms in Eqn. \eqref{eqn:expected-return} have several interpretations. 
First, they could represent the output of some (unknown) market process governing the returns. 
Another actionable interpretation is to think of them as weighting factors. 

Suppose the recent trend in prices for firm $i$ is significantly different from long-term historical price trends, 
e.g., firm $i$ has recently experienced a sustained period of decline. 
You could weigh the recent data more highly to calculate a localized expected return (or volatility) 
that is more representative of current trends. 
Of course, the opposite could be true; you could also empathize with older versus newer data. Many weighting schemes could be used; any approach that obeys the axioms of probability will work!
However, let's borrow a strategy from chemical physics; namely, we'll assume $p(t)$ follows the 
\href{https://en.wikipedia.org/wiki/Boltzmann_distribution}{Boltzmann distribution}:

\begin{definition}[Boltzmann weighted excess returns]\label{defn-bwer}
Let the probability factors $p(t)$ follow a \href{https://en.wikipedia.org/wiki/Boltzmann_distribution}{Boltzmann distribution}:
\begin{equation*}\label{eqn-boltzmann-weight}
    p(t) = \frac{1}{Z}\cdot\exp(-\lambda\cdot\epsilon_{t})
\end{equation*}
where the partition function $Z \equiv \sum_{t=1}^{T}\exp(-\lambda\cdot\epsilon_{t})$.
The parameter $\lambda\geq{0}$ controls the rate of decay, while $\epsilon_{t}>0$ is a pseudo energy of the market at time $t$, 
which we model as $\epsilon_{t} = t$. Then, the Boltzmann weighted expected excess return for the choice of $\epsilon_{t} \sim t$ is given by: 
\begin{equation*}\label{eqn-boltzmann-weight-expectation}
    \mathbb{E}_{B}\left(R_{i}\right) = \frac{1}{Z}\cdot\sum_{t=1}^{T}\exp\left(-\lambda\cdot{t}\right){R}_{i,t}
\end{equation*}
\end{definition}
Depending upon how we choose the decay parameter $\lambda$ and the pseudo energies in \ref{defn-bwer}, we can recover 
equally weighted, past or present exponentially weighted expectations. Algorithm \ref{algo-boltzmann-weighted-excess-returns} 
describes how to calculate the Boltzmann weighted expected excess return for a firm $i$ given a historical data set $\mathcal{D}_{i}$:
\begin{algorithm}[h]
\caption{Boltzmann weighted excess returns}\label{algo-boltzmann-weighted-excess-returns}
\begin{algorithmic}[1]
\State \textbf{Input:} Historical data set $\mathcal{D}_{i}$ for firm $i$, hyperparameter $\lambda\geq{0}$, the time step $\Delta{t}$,
and the risk-free rate $r_{f}$.
\Statex
\Procedure{Boltzmann-Weighted-Returns}{$\mathcal{D}$, $\lambda$, $\Delta{t}$, $r_{f}$}
\Statex
\State $\bar{\mathcal{D}_{i}}\leftarrow\mathcal{D}_{i}$ \Comment{Sort dataset $\mathcal{D}_{i}$ from newest to oldest prices for firm $i$}.
\State initialize $\mathcal{R}~\leftarrow~\mathbf{0}$ \Comment{Initialize the return array as zeros}
\State initialize $W~\leftarrow~\mathbf{0}$ \Comment{Initialize the weight array as zeros}
\Statex
\For{$t=2\rightarrow{T-1}$}
\State $\epsilon_{t} \leftarrow t - 1$ \Comment{Set the pseudo energy}
\State $W(t) \leftarrow \exp(-\lambda\cdot\epsilon_{t})$ \Comment{Calculate the weight factor $W(t)$}
\State{$S_{i,t-1} \gets \bar{\mathcal{D}}_{i}[t-1]$}\Comment{Select the price of stock $i$ at time $t-1$}
\State{$S_{i,t} \gets \bar{\mathcal{D}}_{i}[t]$}\Comment{Select the price of stock $i$ at time $t$}
\State{$\mathcal{R}(t) \gets W(t)\cdot\left(\left(1/\Delta{t}\right)\cdot\ln\left(S_{i,t}/{S_{i,t-1}}\right) - r_{f}\right)$}\Comment{Set $r_{f} = 0$ for regular growth rate}
\EndFor

\Statex
\State $Z \leftarrow \sum_{t=1}^{T}W(t)$ \Comment{Calculate the partition function $Z$}
\State $\mathbb{E}_{B}\left(R\right)\leftarrow(1/Z)\times\sum\mathcal{R}$ \Comment{Calculate the Boltzmann weighted expected return}
\EndProcedure

\end{algorithmic}
\end{algorithm}

\section*{Weighted Variance}
Fill me in.

\section*{Time-Dependent Return and Variance Parameters}
For the standard geometric Brownian motion equations, the drift (growth) coefficient $\mu$ and the diffusion (volatility) coefficient $\sigma^{2}$ are assumed constant. 
However, in many practical cases, this may not be true. For example, macroeconomic conditions may change, leading to different price dynamics. 
Thus, one could imagine modifying the geometric Brownian motion model to capture time-dependent drift and diffusion coefficients.
For example, we could construct a separate model that accounts for how the diffusion (volatility) coefficient $\sigma$ 
changes as a function of time. However, these more sophisticated models do not have quickly developed analytical solutions except in rare cases. 
Thus, we must develop numerical methods to approximate the solutions of these models. There are several approaches to numerically estimating a solution to the time-dependent drift and diffusion coefficient problem.
However, we'll consider the simplest and the most accessible approach, albeit the least accurate, 
namely the Euler-Maruyama (EM) method. 
Alternatively, we could use time-series models, e.g., ARIMA, GARCH, etc., to estimate the time-dependent drift and diffusion coefficients.
Let's start with the EM method.

\subsection*{Euler-Maruyama Method}
The Euler-Maruyama method is a simple numerical method for solving stochastic differential equations (SDEs). 
The \href{https://en.wikipedia.org/wiki/Euler–Maruyama_method}{Euler-Maruyama (EM) method}
is an extension to stochastic differential equations (SDEs) of the \href{https://en.wikipedia.org/wiki/Euler_method}{Euler method}
used for approximating the solution of ordinary differential equations (ODEs). 

\begin{definition}[Euler-Maruyama Method]\label{defn-em-method}
For the general scalar stochastic differential equation describing the random variable $X$ (assuming Wiener noise):
\begin{equation}\label{eqn-generic-scalar-sde}
dX = a(X,t)dt + b(X,t)dW
\end{equation}
where $a(X,t)$ and $b(X,t)$ are functions of the state $X$ and $t$, and $dW$ is the Wiener noise process.
The EM method gives the approximate solution:
\begin{equation*}\label{eqn-em-method}
X_{k+1} = X_{k} + a(X_{k}, t_{k})h + b(X_{k},t_{k})\sqrt{h}Z\left(0,1\right)
\end{equation*}
where the step size $h=t_{k+1} - t_{k}$ and $Z(0,1)$ denotes a standard normal random variable.  
While the EM method is straightforward to implement, it has a poor overall accuracy with error on order $\sqrt{h}$; 
thus, small step sizes are required to get accurate solutions.
\end{definition}

\subsubsection*{Ornstein-Uhlenbeck model}
The \href{https://en.wikipedia.org/wiki/Ornstein–Uhlenbeck_process}{Ornstein-Uhlenbeck model} of mean reversion is a type of Brownian motion model 
with a modified mean-reverting drift term:
\begin{equation*}\label{eqn-ou-model}
dX = \theta\left(\mu-X\right)dt + {\sigma}dW
\end{equation*}
where $\theta>0$ is the time-constant controlling the mean reversion, 
$\mu$ is the long-term price target, and $\sigma>0$ denotes the volatility parameter [REFME]. 
We can use the EM method to discretize the Ornstein-Uhlenbeck model:
\begin{equation*}\label{eqn-EM-soln-OU-model}
X_{k+1} = X_{k} + \theta\left(\mu-X_{k}\right)h + \left(\sigma\sqrt{h}\right)Z(0,1)
\end{equation*}
where $Z(0,1)$ denotes a standard normally distributed random variable, 
and $h = t_{k+1} - t_{k}$ denotes the fixed time step size, where the subsscript $k$ denotes the time index. 

\subsection*{Time-series Models of Time-Dependent Parameters}
Fill me in.

\section*{Summary}
Fill me in. 

\bibliography{References_v1}

\clearpage
\printindex

\end{document}
