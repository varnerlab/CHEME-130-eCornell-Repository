\documentclass[11pt]{article}

% load some asm stuff -
\usepackage{amssymb}
\usepackage{amsmath}
\usepackage{amsthm}
%\usepackage{palatino,lettrine}
\usepackage{fancyhdr}
\usepackage{epsfig}
\usepackage[square,sort,comma,numbers]{natbib}
\usepackage{simplemargins}
\usepackage{setspace}
\usepackage{wrapfig}
\usepackage{hyperref}
%\usepackage{boiboites}
\usepackage[margin=0pt,font=small,labelfont=bf]{caption}
\newcommand{\boldindex}[1]{\textbf{\hyperpage{#1}}}
\usepackage{makeidx}\makeindex
\bibliographystyle{plos2015}

\usepackage{algpseudocode}
\usepackage{algorithm}


% Set the size
%\textwidth = 6.75 in
%\textheight = 9.75 in
%\oddsidemargin = 0.0 in
%\evensidemargin = 0.0 in
%\topmargin = 0.01 in
%\headheight = 0.0 in
%\headsep = 0.25 in
%\parskip = 0.15in
% \doublespace
\setallmargins{1in}

\newtheorem{example}{Example}[section]
\newtheorem{thm}{Theorem}[section]
\newtheorem{property}{Property}[section]

\theoremstyle{definition}
\newtheorem{defn}[thm]{Definition}

\makeatletter
% \renewcommand\subsection{\@startsection
% 	{subsection}{2}{0mm}
% 	{-0.05in}
% 	{0.05\baselineskip}
% 	{\normalfont\normalsize\bfseries}}
\renewcommand\subsubsection{\@startsection
	{subsubsection}{2}{0mm}
	{-0.05in}
	{-0.5\baselineskip}
	{\normalfont\normalsize\itshape\bfseries}}
\renewcommand\paragraph{\@startsection
	{paragraph}{2}{0mm}
	{-0.05in}
	{-0.5\baselineskip}
	{\normalfont\normalsize\itshape}}
\makeatother
\linespread{1.1}

\fancypagestyle{proposal}{\fancyhf{}%
	\fancyhead[RO,LE]{\thepage}%
	\fancyhead[LO,RE]{CHEME 132 Module 2 Single Asset Geometric Brownian Motion}%
	\renewcommand\headrulewidth{1pt}}
\pagestyle{proposal}

\usepackage{mdframed}
\definecolor{lgray}{rgb}{0.92,0.92,0.92}
\definecolor{antiquewhite}{rgb}{0.98,0.92,0.84}
\definecolor{lightskyblue}{rgb}{0.93,0.95,0.99}

% defn environment
\mdfdefinestyle{theoremstyle}{% 
    linecolor=black,linewidth=1pt,% 
    frametitlerule=true,% 
    frametitlebackgroundcolor=lgray, 
    innertopmargin=\topskip,} 
\mdtheorem[style=theoremstyle]{definition}{Definition}

% concept environment
\mdfdefinestyle{conceptstyle}{% 
    linecolor=black,linewidth=1pt,% 
    frametitlerule=true,% 
    frametitlebackgroundcolor=lightskyblue, 
    innertopmargin=\topskip,} 
\mdtheorem[style=conceptstyle]{concept}{Concept}
\newcommand{\newterm}[1]{{\it #1}}

% Single space'd bib -
\setlength\bibsep{0pt}

\renewcommand{\rmdefault}{phv}\renewcommand{\sfdefault}{phv}
%\newboxedtheorem[boxcolor=black, background=gray!5,titlebackground=orange!20,titleboxcolor = black]{color_box_example}{Example}{test}

% Change the number format in the ref list -
\renewcommand{\bibnumfmt}[1]{#1.}

% Change Figure to Fig.
\renewcommand{\figurename}{Fig.}
\usepackage{enumitem}
\setlist{noitemsep} % or \setlist{noitemsep} to leave space around whole list

%Joycelyn Chan, Joshua Lequieu, Michael Paull, Chidanand Balaji, Ryan Tasseff
%Our derivation follows closely the earlier development of Fredrickson \citep{Fredrickson:1976fk}.

% Begin ...
\begin{document}

%\begin{titlepage}
{\par\centering\textbf{\Large CHEME 132 Module 2: Single Asset Geometric Brownian Motion Simulations}}
\vspace{0.2in}
{\par \centering \large{Jeffrey D. Varner}}
\vspace{0.05in}
{\par \centering \large{Smith School of Chemical and Biomolecular Engineering}}
{\par \centering \large{Cornell University, Ithaca NY 14853}}
% \vspace{0.1in}
% {\par \centering \small{Copyright \copyright\ Jeffrey Varner 2018. All Rights Reserved.}}\\

%\end{titlepage}
\date{}
\thispagestyle{empty}

\setcounter{page}{1}

\section*{Introduction}
\href{https://en.wikipedia.org/wiki/Geometric_Brownian_motion}{Geometric Brownian motion (GBM)} is a continuous-time stochastic model in which the random variable $S(t)$, 
e.g., the share price of a firm, is described by a deterministic trajectory corrupted by a \href{https://en.wikipedia.org/wiki/Wiener_process}{Wiener stochastic noise process}:
\begin{equation}\label{eqn:GBM}
\frac{dS}{S} = {\mu}\,dt + \sigma\,{dW}
\end{equation}
The constant $\mu$ denotes a drift parameter, i.e., the growth rate of the share price return, $\sigma$ is a volatility parameter, i.e., 
the dispersion of the return, $dt$ denotes an infinitesimal time step, and $dW$ represents the output of a 
\href{https://en.wikipedia.org/wiki/Wiener_process}{Wiener noise process}.  
Thus, Eqn. \ref{eqn:GBM} is the continuous-time analog of the discrete-time binomial lattice model we developed previously. 
In this module, we will develop analytical solutions to Eqn. \ref{eqn:GBM}, and tools to estimate the parameters $\mu$ and $\sigma$ from historical data. 


\section*{Analytical solution}
Using Ito's lemma, we can formulate an analytical solution to the GBM equation for a single asset.
Ito's Lemma, developed by K. Ito in 1951, is an analog of the Taylor series for stochastic systems.
Let the random variable $X(t)$ be governed by the general stochastic differential equation:
\begin{equation*}
dX = a\left(X(t),t\right)dt + b\left(X(t),t\right)dW(t)
\end{equation*}
where $dW(t)$ is a one-dimensional Wiener process and $a$ and $b$ are functions of $X(t)$ and $t$. 
Let $Y(t) = \phi\left(t,X(t)\right)$ be twice differentiable with respect to $X(t)$, 
and singly differentiable with respect to $t$. Then, $Y(t)$ is governed by the equation:
\begin{equation*}
dY = \left(\frac{\partial{Y}}{\partial{t}}+a\frac{\partial{Y}}{\partial{X}}+\frac{b^{2}}{2}\frac{\partial^{2}{Y}}{\partial{X}^{2}}\right)dt+b\left(\frac{\partial{Y}}{\partial{X}}\right)dW(t)
\end{equation*}
Let $Y = \ln(S)$, $a = \mu\cdot{S}$, and $b = \sigma\cdot{S}$. 
Then, $Y$ is governed by the stochastic differential equation (using Ito's Lemma):
\begin{equation*}
d\ln(S) = \left(\mu - \frac{\sigma^{2}}{2}\right)dt + \sigma\cdot{dW(t)}
\end{equation*}
We integrate both sides of the equation to obtain from $t_{\circ}$ to $t$:
\begin{equation*}
\int_{t_{\circ}}^{t}d\ln(S) = \int_{t_{\circ}}^{t}\left(\mu - \frac{\sigma^{2}}{2}\right)dt + \int_{t_{\circ}}^{t}\sigma\cdot{dW(t)}
\end{equation*}
which gives:
\begin{equation*}
\ln\left(\frac{S_{t}}{S_{\circ}}\right) = \left(\mu - \frac{\sigma^{2}}{2}\right)\left(t - t_{\circ}\right) + \sigma\cdot\sqrt{t-t_{\circ}}\cdot{Z(0,1)}
\end{equation*}
where the noise term makes use of the definition of the integral of a Wiener process.
Finally, we exponentiate both sides of the equation to obtain the analytical solution to the GBM model:
\begin{equation}\label{eqn:analytical-soln-GBM}
S(t) = S_{\circ}\exp\Biggl[\left(\mu-\frac{\sigma^{2}}{2}\right)\left(t - t_{\circ}\right) + (\sigma\sqrt{t-t_{\circ}})\cdot{Z_{t}(0,1)}\Biggr]
\end{equation}
where $S_{\circ}$ denotes the share price at $t_{\circ}$, and $Z_{t}(0,1)$ denotes a 
\href{https://en.wikipedia.org/wiki/Normal_distribution#Standard_normal_distribution}{standard normal random variable} at time $t$.
The expectation and variance of a GBM model is:
\begin{eqnarray*}
\mathbb{E}\left(S_{t}\right) &=& S_{o}\cdot\exp\left(\mu\cdot\Delta{t}\right)\\
\text{Var}\left(S_{t}\right) &=& S_{\circ}^{2}e^{2\mu\cdot\Delta{t}}\left[e^{\sigma^{2}{\Delta{t}}} - 1\right]
\end{eqnarray*}

\section*{Model parameters}
\subsection*{Estimating the growth parameter $\mu$}
Let $\mathbf{A}$ denote a $\mathcal{S}\times{2}$ matrix, where each row corresponds to a time value. 
The first column of $\mathbf{A}$ is all 1's while the second column holds the $(t_{k}-t_{\circ})$ values. 
Further, let $\mathbf{Y}$ denote the $\ln$ of the share price values (in the same order as the $\mathbf{A}$ matrix). 
Then, the y-intercept and slope (drift parameter) can be estimated by solving the overdetermined system of equations:
\begin{equation*}
\mathbf{A}\mathbf{\theta} + \mathbf{\epsilon} = \mathbf{Y}
\end{equation*}
where $\mathbf{\theta}$ denotes the vector of unknown parameters. 
This system can be solved as:
\begin{equation*}
\mathbf{\theta} = (\mathbf{A}^{T}\mathbf{A})^{-1}\mathbf{A}^{T}\mathbf{Y} - (\mathbf{A}^{T}\mathbf{A})^{-1}\mathbf{A}^{T}\mathbf{\epsilon}
\end{equation*}
where $\mathbf{A}^{T}$ denotes the transpose of the matrix $\mathbf{A}$, and $(\mathbf{A}^{T}\mathbf{A})^{-1}$ denotes the inverse of the square matrix product $\mathbf{A}^{T}\mathbf{A}$. 
Finally, we can estimate the error term $\mathbf{\epsilon}$ by calculating the residuals:
\begin{equation*}
\mathbf{\epsilon} = \mathbf{Y} - \mathbf{A}\mathbf{\theta}
\end{equation*}
and then fitting a normal distribution to the residuals to compute the uncertainty in the estimate of the mean of the drift parameter $\hat{\mu}$. 

\subsection*{Estimating the volatility parameter $\sigma$}


\section*{Summary}
Fill me in.

\bibliography{References_v1}

\clearpage
\printindex

\end{document}
