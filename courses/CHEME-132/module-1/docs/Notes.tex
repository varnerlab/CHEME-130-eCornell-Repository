\documentclass[11pt]{article}

% load some asm stuff -
\usepackage{amssymb}
\usepackage{amsmath}
\usepackage{amsthm}
%\usepackage{palatino,lettrine}
\usepackage{fancyhdr}
\usepackage{epsfig}
\usepackage[round,comma,sort]{natbib}
\usepackage{simplemargins}
\usepackage{setspace}
\usepackage{wrapfig}
\usepackage{hyperref}
%\usepackage{boiboites}
\usepackage[margin=0pt,font=small,labelfont=bf]{caption}
\newcommand{\boldindex}[1]{\textbf{\hyperpage{#1}}}
\usepackage{makeidx}\makeindex
\bibliographystyle{plos2015}

% Set the size
%\textwidth = 6.75 in
%\textheight = 9.75 in
%\oddsidemargin = 0.0 in
%\evensidemargin = 0.0 in
%\topmargin = 0.01 in
%\headheight = 0.0 in
%\headsep = 0.25 in
%\parskip = 0.15in
% \doublespace
\setallmargins{1in}

\newtheorem{example}{Example}[section]
\newtheorem{thm}{Theorem}[section]
\newtheorem{property}{Property}[section]

\theoremstyle{definition}
\newtheorem{defn}[thm]{Definition}

\makeatletter
\renewcommand\subsection{\@startsection
	{subsection}{2}{0mm}
	{-0.05in}
	{-0.5\baselineskip}
	{\normalfont\normalsize\bfseries}}
\renewcommand\subsubsection{\@startsection
	{subsubsection}{2}{0mm}
	{-0.05in}
	{-0.5\baselineskip}
	{\normalfont\normalsize\itshape\bfseries}}
\renewcommand\paragraph{\@startsection
	{paragraph}{2}{0mm}
	{-0.05in}
	{-0.5\baselineskip}
	{\normalfont\normalsize\itshape}}
\makeatother
\linespread{1.1}

\fancypagestyle{proposal}{\fancyhf{}%
	\fancyhead[RO,LE]{\thepage}%
	\fancyhead[LO,RE]{CHEME 132 Module 1 Binomial Models of Equity Prices}%
	\renewcommand\headrulewidth{1pt}}
\pagestyle{proposal}

\usepackage{mdframed}
\definecolor{lgray}{rgb}{0.92,0.92,0.92}
\definecolor{antiquewhite}{rgb}{0.98,0.92,0.84}
\definecolor{lightskyblue}{rgb}{0.93,0.95,0.99}

% defn environment
\mdfdefinestyle{theoremstyle}{% 
    linecolor=black,linewidth=1pt,% 
    frametitlerule=true,% 
    frametitlebackgroundcolor=lgray, 
    innertopmargin=\topskip,} 
\mdtheorem[style=theoremstyle]{definition}{Definition}

% concept environment
\mdfdefinestyle{conceptstyle}{% 
    linecolor=black,linewidth=1pt,% 
    frametitlerule=true,% 
    frametitlebackgroundcolor=lightskyblue, 
    innertopmargin=\topskip,} 
\mdtheorem[style=conceptstyle]{concept}{Concept}
\newcommand{\newterm}[1]{{\it #1}}

% Single space'd bib -
\setlength\bibsep{0pt}

\renewcommand{\rmdefault}{phv}\renewcommand{\sfdefault}{phv}
%\newboxedtheorem[boxcolor=black, background=gray!5,titlebackground=orange!20,titleboxcolor = black]{color_box_example}{Example}{test}

% Change the number format in the ref list -
\renewcommand{\bibnumfmt}[1]{#1.}

% Change Figure to Fig.
\renewcommand{\figurename}{Fig.}
\usepackage{enumitem}
\setlist{noitemsep} % or \setlist{noitemsep} to leave space around whole list

%Joycelyn Chan, Joshua Lequieu, Michael Paull, Chidanand Balaji, Ryan Tasseff
%Our derivation follows closely the earlier development of Fredrickson \citep{Fredrickson:1976fk}.

% Begin ...
\begin{document}

%\begin{titlepage}
{\par\centering\textbf{\Large CHEME 132 Module 1: Lattice Models of Equity Share Price}}
\vspace{0.2in}
{\par \centering \large{Jeffrey D. Varner}}
\vspace{0.05in}
{\par \centering \large{Smith School of Chemical and Biomolecular Engineering}}
{\par \centering \large{Cornell University, Ithaca NY 14853}}
% \vspace{0.1in}
% {\par \centering \small{Copyright \copyright\ Jeffrey Varner 2018. All Rights Reserved.}}\\

%\end{titlepage}
\date{}
\thispagestyle{empty}

\setcounter{page}{1}

% \begin{mdframed}[backgroundcolor=lgray]

% 	\subsection*{Background}
% 	We have discussed idealized reversible power generation and refrigeration cycles, and considered the impact of
% 	process irreversibility. In this lecture module, we will expand on the topic of irreversibility. In particular, we will develop expressions for
% 	the rate of \textit{lost work} caused by irreversibility in terms if the rate of entropy generation and process unit efficiencies.

% 	\vspace{0.1in}
% 	\subsection*{Student outcomes}
% 	At the end of this lecture module, students will be able to:
% 	\begin{itemize}
% 	  \item[O$_1$]{Describe the terms in the entropy balance for an open time dependent and steady-state system}
% 		\item[O$_2$]{Relate the rate of lost work $\dot{W}_{lost}$ to the rate of entropy generation $\dot{S}_{G}$ in a steady-state system.}
% 		\item[O$_3$]{Relate the efficiency of common equipment, e.g., pumps, compressors turbines etc to the rate of entropy generation $\dot{S}_{G}$ in a steady-state system.}
% 	\end{itemize}

% \end{mdframed}

% \clearpage

\section*{Introduction}
A lattice model discretizes the potential future states of the world into a finite number of options. 
For instance, a binomial lattice model has two future states: \texttt{up} and \texttt{down}, while a ternary lattice model has three: \texttt{up}, \texttt{down}, and \texttt{flat}. 
To make predictions, we must assign values and probabilities to each of these future states and then calculate the expected value and variance of future values. 
Thus, we do not know quantities such as share price exactly because we are projecting into the future. Instead, we have only a probabilistic model of the possible future values. 
Let’s look at the simplest possible lattice model, a binomial lattice. This is our first example of a \href{https://en.wikipedia.org/wiki/Stochastic_proces}{stochastic model}(s) of equity pricing.


\clearpage
\printindex

\end{document}
