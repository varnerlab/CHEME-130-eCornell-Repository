\documentclass[11pt]{article}

% load some asm stuff -
\usepackage{amssymb}
\usepackage{amsmath}
\usepackage{amsthm}
%\usepackage{palatino,lettrine}
\usepackage{fancyhdr}
\usepackage{epsfig}
\usepackage[round,comma,sort]{natbib}
\usepackage{simplemargins}
\usepackage{setspace}
\usepackage{wrapfig}
\usepackage{hyperref}
%\usepackage{boiboites}
\usepackage[margin=0pt,font=small,labelfont=bf]{caption}
\newcommand{\boldindex}[1]{\textbf{\hyperpage{#1}}}
\usepackage{makeidx}\makeindex
\bibliographystyle{plos2015}

\usepackage{algpseudocode}
\usepackage{algorithm}


% Set the size
%\textwidth = 6.75 in
%\textheight = 9.75 in
%\oddsidemargin = 0.0 in
%\evensidemargin = 0.0 in
%\topmargin = 0.01 in
%\headheight = 0.0 in
%\headsep = 0.25 in
%\parskip = 0.15in
% \doublespace
\setallmargins{1in}

\newtheorem{example}{Example}[section]
\newtheorem{thm}{Theorem}[section]
\newtheorem{property}{Property}[section]

\theoremstyle{definition}
\newtheorem{defn}[thm]{Definition}

\makeatletter
% \renewcommand\subsection{\@startsection
% 	{subsection}{2}{0mm}
% 	{-0.05in}
% 	{0.05\baselineskip}
% 	{\normalfont\normalsize\bfseries}}
\renewcommand\subsubsection{\@startsection
	{subsubsection}{2}{0mm}
	{-0.05in}
	{-0.5\baselineskip}
	{\normalfont\normalsize\itshape\bfseries}}
\renewcommand\paragraph{\@startsection
	{paragraph}{2}{0mm}
	{-0.05in}
	{-0.5\baselineskip}
	{\normalfont\normalsize\itshape}}
\makeatother
\linespread{1.1}

\fancypagestyle{proposal}{\fancyhf{}%
	\fancyhead[RO,LE]{\thepage}%
	\fancyhead[LO,RE]{CHEME 132 Module 1 Binomial Models of Equity Prices}%
	\renewcommand\headrulewidth{1pt}}
\pagestyle{proposal}

\usepackage{mdframed}
\definecolor{lgray}{rgb}{0.92,0.92,0.92}
\definecolor{antiquewhite}{rgb}{0.98,0.92,0.84}
\definecolor{lightskyblue}{rgb}{0.93,0.95,0.99}

% defn environment
\mdfdefinestyle{theoremstyle}{% 
    linecolor=black,linewidth=1pt,% 
    frametitlerule=true,% 
    frametitlebackgroundcolor=lgray, 
    innertopmargin=\topskip,} 
\mdtheorem[style=theoremstyle]{definition}{Definition}

% concept environment
\mdfdefinestyle{conceptstyle}{% 
    linecolor=black,linewidth=1pt,% 
    frametitlerule=true,% 
    frametitlebackgroundcolor=lightskyblue, 
    innertopmargin=\topskip,} 
\mdtheorem[style=conceptstyle]{concept}{Concept}
\newcommand{\newterm}[1]{{\it #1}}

% Single space'd bib -
\setlength\bibsep{0pt}

\renewcommand{\rmdefault}{phv}\renewcommand{\sfdefault}{phv}
%\newboxedtheorem[boxcolor=black, background=gray!5,titlebackground=orange!20,titleboxcolor = black]{color_box_example}{Example}{test}

% Change the number format in the ref list -
\renewcommand{\bibnumfmt}[1]{#1.}

% Change Figure to Fig.
\renewcommand{\figurename}{Fig.}
\usepackage{enumitem}
\setlist{noitemsep} % or \setlist{noitemsep} to leave space around whole list

%Joycelyn Chan, Joshua Lequieu, Michael Paull, Chidanand Balaji, Ryan Tasseff
%Our derivation follows closely the earlier development of Fredrickson \citep{Fredrickson:1976fk}.

% Begin ...
\begin{document}

%\begin{titlepage}
{\par\centering\textbf{\Large CHEME 132 Module 1: Lattice Models of Equity Share Price}}
\vspace{0.2in}
{\par \centering \large{Jeffrey D. Varner}}
\vspace{0.05in}
{\par \centering \large{Smith School of Chemical and Biomolecular Engineering}}
{\par \centering \large{Cornell University, Ithaca NY 14853}}
% \vspace{0.1in}
% {\par \centering \small{Copyright \copyright\ Jeffrey Varner 2018. All Rights Reserved.}}\\

%\end{titlepage}
\date{}
\thispagestyle{empty}

\setcounter{page}{1}

% \begin{mdframed}[backgroundcolor=lgray]

% 	\subsection*{Background}
% 	We have discussed idealized reversible power generation and refrigeration cycles, and considered the impact of
% 	process irreversibility. In this lecture module, we will expand on the topic of irreversibility. In particular, we will develop expressions for
% 	the rate of \textit{lost work} caused by irreversibility in terms if the rate of entropy generation and process unit efficiencies.

% 	\vspace{0.1in}
% 	\subsection*{Student outcomes}
% 	At the end of this lecture module, students will be able to:
% 	\begin{itemize}
% 	  \item[O$_1$]{Describe the terms in the entropy balance for an open time dependent and steady-state system}
% 		\item[O$_2$]{Relate the rate of lost work $\dot{W}_{lost}$ to the rate of entropy generation $\dot{S}_{G}$ in a steady-state system.}
% 		\item[O$_3$]{Relate the efficiency of common equipment, e.g., pumps, compressors turbines etc to the rate of entropy generation $\dot{S}_{G}$ in a steady-state system.}
% 	\end{itemize}

% \end{mdframed}

% \clearpage

\section*{Introduction}
A lattice model discretizes the potential future states of the world into a finite number of options. 
For instance, a binomial lattice model has two future states: \texttt{up} and \texttt{down}, while a ternary lattice model has three: \texttt{up}, \texttt{down}, and \texttt{flat}. 
To make predictions, we must assign values and probabilities to each of these future states and then calculate the expected value and variance of future values. 
Thus, we do not know quantities such as share price exactly because we are projecting into the future. Instead, we have only a probabilistic model of the possible future values. 
We'll begin with the simplest possible lattice model, a binomial lattice (Fig. \ref{fig:binomial-lattice-schematic}).
\begin{figure}[h]
    \centering
    \includegraphics[width=0.85\textwidth]{./figs/Fig-Binomial-LatticeModels-Schematic.pdf}
    \caption{Binomial lattice model schematic. 
	At each node, the share price can either go \texttt{up} by a factor of $u$ or \texttt{down} by a factor of $d$. 
	The probability of going \texttt{up} is $p$ and the probability of going \texttt{down} is $1-p$. 
	\textbf{A}: Single time-step lookahead.
	\textbf{B}: Two time-step lookahead.
	\textbf{C}: Three time-step lookahead.
	At level of the tree $l$, the potential share price can take on $l+1$ values.
	}\label{fig:binomial-lattice-schematic}
\end{figure}

Let's start with a single time-step lookahead, where we have two possible future states (Fig. \ref{fig:binomial-lattice-schematic}A).
Let the initial share price at time \texttt{0} be $S_{\circ}$ and the share price at future time \texttt{1} be $S_{1}$.
During the transition from time \texttt{0}$\rightarrow$\texttt{1} the world transitions from the current state, to one of two possible future states: \texttt{up} or \texttt{down}.
We move to the \texttt{up} state with probability $p$ or the \texttt{down} state with probability $(1-p)$.
Thus, at the time \texttt{1}, the share price $S_{1}$ can take on one of two possible values: $S^{u} = u\cdot{S_{\circ}}$ if the world moves 
to the \texttt{up} state, or $S^{d} = d\cdot{S_{\circ}}$ if the world moves to the \texttt{down} state. 
As we move to the future, we can continue to build out the lattice model by adding additional time-steps, 
for example consider a two-step ahead prediction (Fig. \ref{fig:binomial-lattice-schematic}B). 
At time \texttt{2}, the share price can take on one of three possible values: $S^{uu} = u^{2}\cdot{S_{\circ}}$ if the world moves
to the \texttt{up-up} state, $S^{ud} = ud\cdot{S_{\circ}}$ if the world moves to the \texttt{up-down} state, or $S^{dd} = d^{2}\cdot{S_{\circ}}$ if the world moves to the \texttt{down-down} state.
We can continue to build out the lattice model by adding additional time-steps, for example consider a three-step ahead prediction (Fig. \ref{fig:binomial-lattice-schematic}C).

\section*{Binomial Lattice Analytical Solution}
Let's consider a binomial lattice model with $n$ time-steps. At each time-step, the share price can either go \texttt{up} by a factor of $u$ or \texttt{down} by a factor of $d$.
Then, at time \texttt{n}, the share price can take on $n+1$ possible values: 
\begin{equation}
	S_{n} = S_{\circ}\times{D_{1}}\times{D_{2}}\times{D_{3}}\times\cdots\times{D_{n}}
\end{equation}
where $D_{i}$ is a random variable that can take on one of two values: $u$ or $d$, 
with probabilities $p$ and $(1-p)$ respectively. 
Thus, at each time-step, the world flips a coin and lands in either the \texttt{up} state with probability $p$ or the \texttt{down} state with probability $(1-p)$.
For a single time-step, we model this random process as a Bernoulli trial, where the probability of success is $p$ and the probability of failure is $(1-p)$.
As the number of time-steps increases we have a series of Bernoulli trials, which is a binomial distribution (Defn: \ref{defn-binomial-distribution}):

\begin{definition}[Binomial Share Price and Probability]\label{defn-binomial-distribution}
	Let $S_{\circ}$ denote the current share price at t = \texttt{0}, $u$ and $d$ denote the \texttt{up} and \texttt{down} factors, 
	and $p$ denote the probability of going \texttt{up}.
	At time $t$, the binomial lattice model predicts the share price $S_{t}$ is given by:
	\begin{equation*}
	S_{t} = S_{\circ}\cdot{u}^{t-k}\cdot{d}^{k}\qquad\text{for}\quad{k=0,1,\dots,t}
	\end{equation*}
	The probability that the share price takes on a partiuclar value at time $t$ is given by:
	\begin{equation*}
	P(S_{t} = S_{\circ}\cdot{u}^{t-k}\cdot{d}^{k}) = \binom{t}{k}\cdot{(1-p)}^{k}\cdot{p}^{t-k}\qquad\text{for}\quad{k=0,1,\dots,t}
	\end{equation*}
	where $\binom{t}{k}$ denotes the binomial coefficient.
\end{definition}

\section*{Models of $u$, $d$ and $p$}
The \texttt{up} and \texttt{down} factors $u$ and $d$, and the probability $p$ can be defined in various ways.  
For example, we can estimate them from historical data, or we can propose models for their values. 

\subsection*{Historical data}
Suppose we have a historical share price dataset from time $1,\dots,T$ for some ticker $j$ denoted as $\mathcal{D}_{j} = \left\{S^{(j)}_{1},S^{(j)}_{2},\dots,S^{(j)}_{T}\right\}$, 
where $S^{(j)}_{i}$ denotes the share price of ticker $j$ at time $i$.
We can use different values for the share price, e.g., the opening price, closing price, high price, low price, etc. 
In our case, when dealing with historical data we will typically use the volume weighted average price (VWAP) for the time period, 
e.g., the VWAP for the day, week, month, etc. Over the time range of the dataset $\mathcal{D}_{j}$, we can calculate the number of \texttt{up} and \texttt{down} moves
occuring between time $i-1$ and $i$, and the magnitude of these moves. 
Then, the fraction of \texttt{up} moves is an estimate of the probability $p$, 
while some measured of the magnitude of the \texttt{up} and \texttt{down} moves, e.g., the average value are estimates of $u$ and $d$ respectively.

\subsubsection*{Estimating $u$, $d$ and $p$ from historical data.}
Suppose we assume the share price of ticker $j$ is continously compounded with an instanteous discount (interest) rate of 
$r^{(j)}_{i,i-1}\equiv\mu^{(j)}_{i,i-1}\cdot\Delta{t}$, i.e., we split the return into a growth rate $\mu^{(j)}_{i,i-1}$ and a time step size $\Delta{t}$.
Then, the share price at time $i$ is goverenved by an expression of the form:
\begin{equation}\label{eqn:share-price-growth-rate}
S^{(j)}_{i} = \exp\left(\mu_{i,i-1}\cdot\Delta{t}\right)\cdot{S^{(i)}_{i-1}}
\end{equation}
where $\mu^{(j)}_{i,i-1}$ denotes the \textit{growth rate} (units: 1/time) for ticker $j$, and $\Delta{t}$ (units: time) 
is the time step size during the time period $(i-1)\rightarrow{i}$.
Solving for the growth rate (and dropping the ticker $j$ superscript for simplicity) gives:
\begin{equation}
\mu_{i,i-1} = \left(\frac{1}{\Delta{t}}\right)\cdot\ln\left(\frac{S_{i}}{S_{i-1}}\right)
\end{equation}
We'll often use daily price data; thus, the natural time frame between $S_{i-1}$ and $S_{i}$ is a single trading day. 
However, subsequently, it will be easier to use an annualized value for the $\mu$ parameter; thus, we let $\Delta{t} = 1/252$, 
i.e., the fraction of an average trading year that occurs in a single trading day; thus, our base time will be years.
We compute the growth rate for each trading day $i$ in a collection of datasets $\mathcal{D}$ using Algorithm \ref{algo-log-return-distributions-equity}.
\begin{algorithm}[h]
    \caption{Logarithmic excess growth rate}\label{algo-log-return-distributions-equity}
    \begin{algorithmic}[1]
        
        \Require Collection of price datasets $\mathcal{D}$, where $\mathcal{D}_{j}\in\mathcal{D}$. All datasets have the same length $N$.	
		\Require list of stocks $\mathcal{L}$, where $\dim\mathcal{L} = M$.
        \Require time step size $\Delta{t}$ between $t$ and $t-1$ (units: years), 
        \Require risk-free rate $r_{f}$ (units: inverse years).

		\Statex
		\State{$M\leftarrow\text{length}(\mathcal{D}_{1})$}\Comment{Number of trading days in the dataset $\mathcal{D}_{j}$}
		\State{$N\leftarrow\text{length}(\mathcal{L})$}\Comment{Number of stocks in the dataset $\mathcal{D}$}
		\State{$\mu\leftarrow\text{Array}(M-1,N)$}\Comment{Initialize empty array of growth rates}
     
        \Statex
        \For{$i\in\mathcal{L}$}
			\State{$\mathcal{D}_{i} \gets \mathcal{D}[i]$}\Comment{Get dataset for stock $i$}

            \For{$t=2\rightarrow{N}$}
				\State{$S_{1} \gets \text{VWAP}(\mathcal{D}_{i}[t-1])$}\Comment{Get volume weighted average price for stock $i$ at time $t-1$}
				\State{$S_{2} \gets \text{VWAP}(\mathcal{D}_{i}[t])$}\Comment{Get volume weighted average price for stock $i$ at time $t$}
                
				\State{$\mu[t-1,i] \gets \left(\frac{1}{\Delta{t}}\right)\cdot\ln\left(\frac{S_{2}}{S_{1}}\right) - r_{f}$}\Comment{Set $r_{f} = 0$ for regular growth rate}
            \EndFor
        \EndFor
	
    \end{algorithmic}
\end{algorithm}
Then we can estimate the \texttt{up} and \texttt{down} factors $u$ and $d$ from the growth rate array 
generated using something like Algorithm \ref{algo-ud-estimation-equity}. 
\begin{algorithm}[h]
	\caption{Estimating $u$, $d$ and $p$ from the $\mu$-array}\label{algo-ud-estimation-equity}
	\begin{algorithmic}[1]

		\Require $\mu$-array from Algorithm \ref{algo-log-return-distributions-equity}.

		\Statex
		\State{$\mu\leftarrow\text{sort}(\mu_{i,i-1})$}
		\State{$N\leftarrow\text{length}(\mu)$}\Comment{Number of growth rates}
		
		\Statex
   	 	\State{$i_{+}\leftarrow \text{findall}(\mu>0)$}\Comment{Find the indices of all \texttt{positive} growth rates}
    	\For{$i\in {i_{+}}$}
        	\State{$\mu[i]\rightarrow(\mu\rightarrow\text{push!}(\text{up}, \exp(\mu\cdot{\Delta{t}})))$}\Comment{Push the \texttt{positive} return $\mu\cdot\Delta{t}$ onto $\text{up}$-array}
    	\EndFor
		\State{$u\leftarrow\text{mean}(\texttt{up})$}\Comment{mean is our estimate of the \texttt{up} factor $u$}
    	
		\Statex
   	 	\State{$i_{-}\leftarrow \text{findall}(\mu<0)$}\Comment{Find the indices of all \texttt{negative} growth rates}
    	\For{$i\in {i_{i}}$}
        	\State{$\mu[i]\rightarrow(\mu\rightarrow\text{push!}(\text{down}, \exp(\mu\cdot{\Delta{t}})))$}\Comment{Push the \texttt{negative} return $\mu\cdot\Delta{t}$ onto $\text{down}$-array}
    	\EndFor
		\State{$d\leftarrow\text{mean}(\texttt{down})$}\Comment{mean is our estimate of the \texttt{down} factor $d$}

		\Statex
		\State{$N_{+} \leftarrow\text{length}(i_{+})$}\Comment{Number of \texttt{positive} growth rates}
		\State{$p\leftarrow N_{+}/N$}\Comment{Estimate of the probability $p$}

		\Statex
		\Return{$u$, $d$ and $p$}
	\end{algorithmic}
\end{algorithm}
While this strategy is simple, it may not be robust. For example, if the dataset $\mathcal{D}_{j}$ is short, i.e., only a few trading days,
then the number of \texttt{up} and \texttt{down} moves will be small, and the estimates of $u$, $d$ and $p$ will be poor.
If a precise estimate of $u$, $d$ and $p$ is required, the number of trading days in the dataset $\mathcal{D}_{j}$ should be large.
Furthermore, the estimates of $u$, $d$ and $p$ are not robust to outliers in the dataset $\mathcal{D}_{j}$.
Thus, we may want to consider other models for computing $u$, $d$ and $p$.

\subsection*{Risk-neutral probability $q$.}
Another approach to compute the parameters in the lattrice is to use the risk-neutral probability $q$. 
This a hypothetical probability that is used to price derivatives (as we shall see later), 
but we could also think of it as a tool to compute a \textit{fair} price for a share of stock.
Suppose we rewrite Eqn. \eqref{eqn:share-price-growth-rate} as:
\begin{equation*}
    \mathcal{D}_{1,0}(\bar{r})\cdot{S_{\circ}} = \mathbb{E}_{\mathbb{Q}}\left(S_{1}\right)
\end{equation*}
where $\mathcal{D}_{1,0}(\bar{r})$ is the continuous discount factor between period $0\rightarrow{1}$, 
and $\bar{r}$ is the effective (constant) risk-free rate. Thus, unlike the previous case, where the share price $S_{i}$ was discounted by the
return $\mu_{i,i-1}\cdot\Delta{t}$ (which could vary in time), here we discount the share price $S_{i}$ by an effective (constant) risk-free rate $\bar{r}$.
The expectation operator $\mathbb{E}_{\mathbb{Q}}(\dots)$ is taken with respect to a \textit{risk neutral probability measure} $\mathbb{Q}$.
Thus, the expectation operator $\mathbb{E}_{\mathbb{Q}}(\dots)$ is:
\begin{equation}\label{eqn:expectation-operator-risk-neutral}
\mathcal{D}_{1,0}(\bar{r})\cdot{S_{\circ}} = q\cdot{S^{u}} + (1-q)\cdot{S^{d}}
\end{equation}
where $q$ is the risk neutral probability of the \texttt{up} state, and $S^{u}$ and $S^{d}$ are the share prices in the \texttt{up} and \texttt{down} states respectively.
The share prices in the \texttt{up} and \texttt{down} states are the product of the \texttt{up} factor $u$ (or a \texttt{down} factor $d$) and the initial share price, 
i.e., $S^{u} = u\cdot{S_{\circ}}$ and $S^{d} = d\cdot{S_{\circ}}$. Substiuting these values in Eqn. \eqref{eqn:expectation-operator-risk-neutral} and solving for $q$ gives:
\begin{equation*}
q = \frac{\mathcal{D}_{1,0}(\bar{r}) - d}{u - d}
\end{equation*}
Thus, we can compute the risk-neutral probability $q$ from the \texttt{up} and \texttt{down} factors $u$ and $d$ and the effective risk-free rate $\bar{r}$.
\clearpage
\printindex

\end{document}
